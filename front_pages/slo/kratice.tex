\chapter*{Seznam uporabljenih kratic}
\thispagestyle{empty}
\begin{table}[ht]
  \centering
  \small
  \setlength{\tabcolsep}{6pt} % malo manjši razmik med stolpci
  \begin{tabularx}{\textwidth}{l|X|X}
    \textbf{kratica} & \textbf{angleško} & \textbf{slovensko} \\ \hline
    \textbf{API}     & Application Programming Interface        & programski vmesnik aplikacije \\
    \textbf{DID}     & Decentralized Identifier                 & decentralizirani identifikator \\
    \textbf{DoS}     & Denial of Service                        & zavrnitev storitve \\
    \textbf{DON}     & Decentralized Oracle Network             & decentralizirano omrežje orakljev \\
    \textbf{DSA}     & Digital Signature Algorithm              & algoritem za digitalno podpisovanje sporočil \\
    \textbf{DSCSA}   & Drug Supply Chain Act                    & zakon o sledenju zdravil, sprejet v ZDA \\
    \textbf{ECDSA}   & Elliptic Curve Digital Signature Algorithm & algoritem eliptičnih krivulj za digitalno podpisovanje \\
    \textbf{EMA}     & European Medicines Agency                & Evropska agencija za zdravila \\
    \textbf{EMVS}     & European Medicines Verification System  & Evropski sistem za sledenje zdravil \\
    \textbf{EWMA}    & Exponentially Weighted Moving Average    & eksponentno uteženo drseče povprečje \\
    \textbf{EVM}     & Ethereum Virtual Machine                 & virtualni stroj Ethereum \\
    \textbf{GDP}     & Good Distribution Practice               & dobra distribucijska praksa \\
    \textbf{GDPR}    & General Data Protection Regulation       & Splošna uredba o varstvu podatkov \\
    \textbf{GMP}     & Good Manufacturing Practice              & dobra proizvodna praksa \\
    \textbf{HSM}     & Hardware Security Module                 & strojni varnostni modul \\ 
  \end{tabularx}
\end{table}

\newpage
% \markboth{}{}
\thispagestyle{empty}
\begin{table}[ht]
  \centering
  \small
  \setlength{\tabcolsep}{6pt} % malo manjši razmik med stolpci
  \begin{tabularx}{\textwidth}{l|X|X}
    \textbf{kratica} & \textbf{angleško} & \textbf{slovensko} \\ \hline
    \textbf{HTTP}    & Hypertext Transfer Protocol              & protokol za prenos hiperteksta \\
    \textbf{HTTPS}   & Hypertext Transfer Protocol Secure       & varni protokol za prenos hiperteksta \\
    \textbf{IoT}     & Internet of Things                       & internet stvari \\
    \textbf{JSON}    & JavaScript Object Notation               & zapis JSON \\
    \textbf{JSON-LD} & JSON for Linking Data                    & JSON za povezane podatke \\
    \textbf{OWASP}   & Open Worldwide Application Security Project & projekt za varnost aplikacij OWASP \\
    \textbf{OWL}     & Web Ontology Language                    & jezik spletnih ontologij OWL \\
    \textbf{PII}     & Personally Identifiable Information      & osebni identifikacijski podatki \\
    \textbf{RPC}     & Remote Procedure Call                    & oddaljeni klic postopkov \\
    \textbf{RDF}     & Resource Description Framework           & ogrodje za opis virov RDF \\
    \textbf{TMS}     & Transport Management System              & sistem za upravljanje transporta \\
    \textbf{URI}     & Uniform Resource Identifier              & enotni identifikator vira \\
    \textbf{VC}      & Verifiable Credential                    & preverljiva poverilnica \\
  \end{tabularx}
\end{table}

\newpage
\chapter*{Seznam uporabljenih pojmov}
\thispagestyle{empty}

\begin{table}[ht]
  \centering
  \small
  \setlength{\tabcolsep}{6pt} % malo manjši razmik med stolpci
  \begin{tabularx}{1\textwidth} { 
  >{\raggedright\arraybackslash}p{0.25\textwidth}|
  >{\raggedright\arraybackslash}X}
    \textbf{pojem} & \textbf{razlaga} \\ \hline \hline
    \textbf{Časovni žig}   & Zaporedje znakov, ki zagotavlja podpis dokumenta z veljavnim digitalnim potrdilom v določenem časovnem trenutku s povezovanjem datuma, časa podpisa in podatkov na kriptografsko varen način. \\ \hline
    \textbf{Digitalni podpis} &
    Matematična shema za preverjanje pristnosti digitalnih dokumentov. Vsebuje tri algoritme: generiranje javnih in zasebnih ključev, podpisni algoritem in algoritem preverjanja podpisa. \\ \hline
    \textbf{Digitalno potrdilo} &
    Strukturiran in kriptografsko podpisan zapis, ki javni ključ poveže z identiteto določenega subjekta izdan s strani zaupanja vrednega izdajatelja. Prejemniku omogoča preverjanje, komu pripada dani javni ključ. \\ \hline
    \textbf{Javni ključ} &
    Del para javnega in zasebnega ključa v asimetrični kriptografiji, namenjen deljenju z drugimi, ki ga uporabljajo za preverjanje digitalnih podpisov imetnika ali za šifriranje sporočil, ki jih lahko dešifrira le ustrezni zasebni ključ. \\ \hline
    \textbf{Proof of Stake} &
    Konsenzni mehanizem, kjer vpliv udeleženca na predlaganje in potrjevanje blokov temelji na količini vloženih (z zaklenitvijo zavarovanih) sredstev v omrežju, pri čemer ekonomski vložek deluje kot spodbuda za pošteno vedenje in varovalka pred napadi. \\

  \end{tabularx}
\end{table}

\newpage
\markboth{}{}
\begin{table}[ht]
  \centering
  \small
  \setlength{\tabcolsep}{6pt} % malo manjši razmik med stolpci
  \begin{tabularx}{1\textwidth} { 
  >{\raggedright\arraybackslash}p{0.25\textwidth}|
  >{\raggedright\arraybackslash}X}
    \textbf{pojem} & \textbf{razlaga} \\ \hline \hline

    \textbf{Proof of Work} &
    Konsenzni mehanizem verig blokov, pri katerem mora vozlišče za predlaganje novega bloka izvesti računsko zahteven izračun. Otežuje ponarejanje blokov in napade z ustvarjanjem velikega števila lažnih blokov. \\ \hline

    \textbf{Veriga blokov} &
    Razpršena, kriptografsko povezana knjižnica digitalnih zapisov (blokov), ki vsebujejo podatke o transakcijah in se povezujejo z zgoščenimi povzetki prejšnjih blokov \\ \hline

    \textbf{Zasebni ključ} &
    Tajni del para javnega in zasebnega ključa, ki ga pozna le imetnik in se uporablja za ustvarjanje digitalnih podpisov ali šifriranje \\ \hline

    \textbf{Zgoščena vrednost} &
    Rezultat zgoščevalne funkcije. \\ \hline

    \textbf{Zgoščevalna funkcija} &
    Matematična funkcija, ki vzame vhodne podatke poljubne velikosti in jih preslika v zgoščeno vrednost fiksne velikosti, pri čemer že majhna sprememba vhodnih podatkov povzroči veliko spremembo zgoščene vrednosti. Pomembna lastnost zgoščevalne funkcije je, da iz izhodne zgoščene vrednosti v doglednem času ne moremo izračunati vhodnih podatkov. \\

  \end{tabularx}
\end{table}