%================================================================
% SLO
%----------------------------------------------------------------
% datoteka: 	thesis_template.tex
%
% opis: 		predloga za pisanje diplomskega dela v formatu LaTeX na
% 				Univerza v Ljubljani, Fakulteti za računalništvo in informatiko
%
% pripravili: 	Matej Kristan, Zoran Bosnić, Andrej Čopar,
%			  	po začetni predlogi Gašperja Fijavža
%
% popravil: 	Domen Rački, Jaka Cikač, Matej Kristan
%
% verzija: 		30. september 2016 (dodan razširjeni povzetek)
%================================================================


%================================================================
% SLO: definiraj strukturo dokumenta
% ENG: define file structure
%================================================================
\documentclass[a4paper, 12pt]{book}


%================================================================
% SLO: Odkomentiraj "\SLOtrue " za izbiro slovenskega jezika
% ENG: Uncomment "\SLOfalse" to chose English languagge
%================================================================
\newif\ifSLO
\newif\ifTRACKEXIST
\newif\ifTRACKCS
\newif\ifPROGRAMMM

% ---------------------------------------------------------------------------------------
% IMPORTANT: Adjust the thesis language, your study program and course within this block
% ---------------------------------------------------------------------------------------
% switch language
\SLOtrue % Enables Slovenian language
%\SLOfalse  % Enables English language

% switch programs: Computer science and Multimedia. Set to false if the program is in Multimedia
\PROGRAMMMfalse
%\PROGRAMMMtrue

% switch on if your program is divided into tracks CS and DS, otherwise leave it false
% CAUTION: if you were first enrolled into your program before school year 2019/2020, your program is not divided into tracks. In any case, be absolutely sure you select the correct variant. IF IN DOUBT, always contact the student office to advise you.
%
\TRACKEXISTfalse
%\TRACKEXISTtrue

% default course name is "Computer science" if your course name is "Data science", set the following switch to false
\TRACKCStrue % uncomment if the thesis is from course "Information science"
%\TRACKCSfalse % uncomment if the thesis is from course "Data Science"
% -------------------------------------------------------------------------------------------
% End of language, program and course adjustment
% -------------------------------------------------------------------------------------------


%================================================================
% SLO: vključi oblikovanje in pakete
% ENG: include design and packages
%================================================================
\input{style/thesis_style}

%----------------------------------------------------------------
% |||||||||||||||||||||| USTREZNO POPRAVI |||||||||||||||||||||||
% |||||||||||||||||||||| EDIT ACCORDINGLY |||||||||||||||||||||||
%----------------------------------------------------------------
\newcommand{\ttitle}{Sistem za zagotavljanje zaupanja v dobavni verigi zdravil z uporabo verige blokov}
\newcommand{\ttitleEn}{Trust System in Pharmaceutical Supply Chain Using Blockchain}
\newcommand{\tsubject}{\ttitle}
\newcommand{\tsubjectEn}{\ttitleEn}
\newcommand{\tauthor}{Žan Pižmoht}
\newcommand{\temail}{zp6881@student.uni-lj.si}
\newcommand{\myyear}{2025}
\newcommand{\tkeywords}{upravljanje zaupanja, tehnologija veriženja blokov, dobavna veriga zdravil, ontologija}
\newcommand{\tkeywordsEn}{trust management, blockchain, pharmaceutical supply chain, ontology}
\newcommand{\mysupervisor}{prof.~dr. Vlado Stankovski}
\newcommand{\mycosupervisor}{doc.~dr.\ Petar Kochovski}

% include formatted front pages
\input{style/thesis_front_pages}

%================================================================
% ENG: main pages of the thesis
%================================================================

% Osnutek magistrske naloge v slovenščini
% (Uporabite v svojem LaTeX slogovnem predlogu)

\chapter{Uvod}
\label{sec:uvod}

Farmacevtske dobavne verige sodijo med najbolj regulirana in podatkovno zahtevna okolja v sodobni industriji. Zdravila, cepiva in drugi farmacevtski proizvodi potujejo med številnimi akterji, kot so proizvajalci, distributerji, transporterji, regulatorji in zdravstvene ustanove. Od vseh se pričakuje visoka stopnja zanesljivosti in skladnosti, vendar so tovrstne verige kljub temu izpostavljene številnim tveganjem. Pogosta so ponarejena zdravila, nepravilno ravnanje z izdelki, nepopolna poročila ter pomanjkanje transparentnosti, kar v veliki meri izhaja iz nezmožnosti pravočasnega preverjanja zaupanja v posamezne akterje.

Obstoječi sistemi za upravljanje zaupanja pogosto temeljijo na delno centraliziranih podatkovnih virih, ročnih revizijah ali statičnih ocenah, ki ne odražajo dovolj hitro dejanskega obnašanja akterjev. Takšni pristopi omejujejo sposobnost regulatorjev in podjetij, da prepoznajo tveganja, ki izhajajo iz sprememb v kakovosti ali skladnosti delovanja. V literaturi so predlagani različni koncepti, ki vključujejo semantično modeliranje, decentralizirane sisteme, kriptografsko preverjanje ter obdelavo telemetrije. Manjka pa splošna zasnova modela, ki bi te elemente povezala v enoten in razširljiv okvir za dinamično ocenjevanje zaupanja.

Magistrsko delo se zato osredotoča na vprašanje, kako v farmacevtski dobavni verigi oblikovati model in mehanizem, ki omogoča transparentno, preverljivo in razširljivo upravljanje zaupanja med akterji, ne da bi bilo treba razkrivati občutljive operativne ali poslovne podatke.

Cilj naloge je razviti konceptualni in arhitekturni okvir ter na njegovi podlagi izdelati prototip sistema, ki omogoča:
\begin{itemize}
  \item formalno predstavitev akterjev, njihovih lastnosti, politik in dokazil,
  \item uporabo več neodvisnih pogledov na zaupanje in različnih kriterijev ocenjevanja,
  \item zbiranje, preverjanje in usklajevanje ocen iz več virov,
  \item zapis končnih odločitev v nespremenljivo in preverljivo podatkovno strukturo.
\end{itemize}

Hipoteza naloge je, da je mogoče tak prototip zasnovati tako, da demonstrira izvedljivost decentraliziranega večpoglednega sistema za izračun zaupanja ter hkrati ohrani lastnosti, ki so ključne za regulativno okolje: razložljivost postopkov, razširljivost arhitekture, odpornost na posamične okvare ter minimizacijo razkritja podatkov.

V nadaljevanju naloga najprej predstavi teoretično ozadje upravljanja zaupanja v razpršenih sistemih ter izzive, značilne za farmacevtske dobavne verige. Sledi opis konceptualnega modela zaupanja in arhitekture sistema, ki omogoča izračun in usklajevanje ocene zaupanja. Nato je opisana implementacija prototipa, sledijo eksperimentalni scenariji, ki prikazujejo praktično delovanje sistema. Zaključek povzame ugotovitve in predstavi ključne smernice za nadaljnjo nadgradnjo rešitve.

\chapter{Pregled literature in ozadje}
\label{ch:background}
To poglavje združuje temeljne koncepte, na katerih temelji sistem, razvit v okviru magistrske naloge. Začnemo z razumevanjem zaupanja in njegovih lastnosti ter s pregledom pristopov, ki se v razpršenih sistemih uporabljajo za njegovo modeliranje in ocenjevanje. V nadaljevanju se posvetimo farmacevtskim dobavnim verigam, kjer izpostavimo ključne težave, ki se pojavljajo pri zagotavljanju sledljivosti, varnosti in sodelovanja med akterji. Tu omenimo obstoječe rešitve, ki se že uporabljajo v praksi in predstavimo probleme, ki jih te rešitve imajo. Sledi pregled tehnologije veriženja blokov, pametnih pogodb in orakljev, ki danes predstavljajo pomemben del modernih decentraliziranih arhitektur. Poglavje zaključimo s predstavitvijo semantičnega spleta in metod, ki so potrebne za formalno opisovanje akterjev in njihovih lastnosti. Ta poglavja skupaj strnemo v množico, v kateri lahko prepoznamo vrzeli obstoječih pristopov in pripravimo teren za rešitev, ki jo predstavimo v nadaljevanju naloge.

\section{Zaupanje}

Pojem zaupanja je temeljni gradnik pri zasnovi našega sistema za farmacevtsko dobavno verigo. Zaupanje je kompleksen, večdimenzionalen in interdisciplinaren koncept, ki ga obravnavajo različne vede. Obravnava se v sociologiji, psihologiji, ekonomiji, računalništvu in še ostalih vedah. Prav zaradi te raznolikosti v literaturi ne obstaja enotna definicija, temveč več pristopov, ki poudarjajo različne vidike \cite{holtmanns2008trust,grandison2000survey}.

\subsection{Definicije zaupanja}
V družboslovnem kontekstu je zaupanje pogosto opredeljeno kot stanje pozitivnih pričakovanj glede dejanj druge osebe v okoliščinah, kjer obstaja določena stopnja tveganja \cite{boon1991interpersonal}. Gambetta \cite{gambetta1988trust} zaupanje definira kot subjektivno verjetnost, da bo agent opravil določeno dejanje, še preden je to mogoče preveriti. Mayer, Davis in Schoorman \cite{mayer1995integrative} pa ga opredelijo kot pripravljenost ene stranke, da se izpostavi ranljivosti glede na dejanja druge stranke, ob pričakovanju, da bo ta delovala v skladu s pričakovanji.

V digitalnem okolju se koncept zaupanja prenaša iz družbenega v tehnični kontekst. Denning \cite{denning1993trust} trdi, da je zaupanje v sistem lastnost, ki jo je mogoče formalno določiti in jo modelirati. Sistem je torej zaupanja vreden, če mu njegovi uporabniki zaupajo glede na opaženo skladnost z vnaprej določenimi standardi. Grandison in Sloman \cite{grandison2000survey} zaupanje obravnavata kot kvalificirano prepričanje zaupnika o kompetentnosti, integriteti, varnosti in zanesljivosti zaupanja vrednega subjekta.

\subsection{Lastnosti zaupanja}

Zaupanje je po svoji naravi večdimenzionalen pojem, ki ga ni mogoče zajeti z eno samo definicijo. V literaturi se ponavlja, da zaupanje vključuje čustvene in vedenjske komponente, ki se med seboj prepletajo in vplivajo na odločanje posameznika ali sistema \cite{mcknight2001trust, karthik2017ontology}. V družbenem kontekstu gre za pripravljenost posameznika, da se izpostavi tveganju na podlagi prepričanja, da bo druga stran delovala predvidljivo in skladno s pričakovanji. V informacijskih sistemih pa to pomeni sposobnost sistema, da sprejema odločitve o sodelovanju na podlagi preteklih izkušenj in ocenjene zanesljivosti entitet.

Zaupanje obstaja le v razmerju med dvema ali več entitetami, kjer ena zaupa drugi v določeni situaciji ali kontekstu. Takšno razmerje je dinamično, saj se lahko stopnja zaupanja sčasoma povečuje ali zmanjšuje. Poleg tega je zaupanje asimetrično: dejstvo, da entiteta A zaupa entiteti B, še ne pomeni, da bo tudi B zaupala A \cite{mayer1995integrative}.

Ena od osrednjih značilnosti zaupanja je njegova povezanost s tveganjem in negotovostjo. Zaupanje je smiselno le v okoliščinah, kjer obstaja možnost, da drugi akter ne bo ravnal skladno s pričakovanji \cite{boon1991interpersonal}. Če bi imeli popoln nadzor ali popolno informacijo, zaupanje sploh ne bi bilo potrebno. V kontekstu varnostnih in distribucijskih sistemov zaupanje dopolnjuje formalne zaščitne mehanizme, kot so avtentikacija in šifriranje, saj omogoča oceno vedenja akterjev tam, kjer tehnični ukrepi ne zadostujejo \cite{denning1993trust}.

Zaupanje je tudi subjektivno in kontekstualno. Različni akterji lahko enake vedenjske vzorce interpretirajo različno, odvisno od svojih ciljev, regulatornih zahtev, izkušenj, prepričanj itn.

\subsection{Atributi zaupanja}
Da bi bilo zaupanje uporabno v digitalnem okolju, ga je potrebno izraziti z merljivimi atributi, ki opisujejo lastnosti ali vedenje zaupanja vrednega subjekta. Ti atributi omogočajo, da se konceptualna ideja zaupanja preslika v formalne modele. V literaturi se pojavljajo različne skupine atributov.

Klasični modeli, kot sta Mayer, Davis in Schoorman \cite{mayer1995integrative} ter McKnight in Chervany \cite{mcknight2001trust}, opredeljujejo tri temeljne dimenzije zaupanja: kompetentnost, integriteto in dobrohotnost. Kompetentnost se nanaša na sposobnost entitete, da učinkovito izvede nalogo, integriteta na njeno zavezanost etičnim in profesionalnim načelom, dobrohotnost pa na nenamernost povzročanja škode. Kasneje so bili tem trem dodani še predvidljivost in zanesljivost, ki poudarjata konsistentnost vedenja skozi čas.

V digitalnih in porazdeljenih okoljih so se pojavili dodatni atributi, ki izhajajo iz tehničnih vidikov zaupanja. Grandison in Sloman \cite{grandison2000survey} zaupanje opišeta kot prepričanje o pristojnosti, varnosti in verodostojnosti drugega subjekta. Denning \cite{denning1993trust} poudarja, da je zaupanje sistema mogoče ocenjevati tudi z vidika varnosti, skladnosti s pravili ter preverljivostjo delovanja. Ti atributi omogočajo formalno ocenjevanje zaupanja med akterji.

V farmacevtski dobavni verigi imajo atributi zaupanja izrazito regulatorno in sledilno komponento. Uddin in sodelavci \cite{uddin2021blockchain} poudarjajo, da so pri zagotavljanju zaupanja ključni kazalniki sledljivost, pristnost in celovitost podatkov.
Kayhan in sodelavci \cite{kayhan2022ensuring} izpostavljajo še pomen transparentnosti, nespremenljivosti in izvorne sledljivosti (angl. provenance), ki jih omogoča tehnologija veriženja blokov.

Predstavljeni atributi ne predstavljajo univerzalne množice meril, temveč zbirko značilnosti, ki jih lahko posamezen akter uporabi pri lastni presoji zaupanja. V sistemu, ki je predstavljen v tej nalogi, se atributi uporabljajo predvsem kot podatkovne lastnosti (angl. data properties) v ontologiji. Te omogočajo, da akterji formalno zapišejo svoje kriterije za ocenjevanje drugih entitet. S tem v sistemu omogočamo standardizirano predstavitev vhodnih metrik, ki jih posamezni akterji uporabljajo pri svojem algoritmu za izračunavanje zaupanja.

\section{Upravljanje zaupanja v razpršenih sistemih}
\label{sec:trust-management-distributed-systems}
Huaizhi Li in Mukesh Singhal v članku \textit{Trust Management in Distributed Systems} \cite{li2007survey} opredelita upravljanje zaupanja kot proces zbiranja informacij, potrebnih za vzpostavitev zaupanja med entitetami, kot tudi dinamičnega spremljanja in prilagajanja obstoječih razmerij zaupanja. Avtorja poudarjata, da v razpršenih okoljih kot so internet, sistemi enakovrednih entitet (angl. \texttt{peer-to-peer networks}) in mobilna omrežja ad hoc pogosto sodelujejo entitete, ki se med seboj ne poznajo. Zato zaupanje postane ključni mehanizem za zmanjšanje tveganja in zagotavljanje sodelovanja. Namen sistema za upravljanje zaupanja je ohranjati ažurne in skladne informacije o zaupanju med akterji v omrežju. V naslednjih podpoglavjih povzemamo različne pristope k upravljanju zaupanja, ki so jih avtorji predstavili v svojem pregledu literature.

\subsection{Dokazni modeli zaupanja}
Dokazni modeli temeljijo na preverljivih dokazih, kot so digitalna potrdila, javni ključi ali kriptografski podpisi.
Ti pristopi zagotavljajo preverjanje identitete in integritete entitet, ne pa tudi njihove dejanske zanesljivosti.
Najpogostejša primera sta hierarhični sistem X.509 \cite{housley1999rfc2459} in decentralizirani sistem PGP \cite{elkins2001mime}, kjer se zaupanje posreduje prek verige certifikatov oziroma prek subjektivnih ocen uporabnikov.
Tak model se uporablja predvsem pri inicializaciji zaupanja in zagotavlja osnovno preverjanje pristnosti brez ocenjevanja vedenja.

\subsection{Priporočilni modeli zaupanja}
Priporočilni modeli gradijo zaupanje na osnovi izkušenj in posredovanih ocen drugih entitet.
Vsaka entiteta ocenjuje druge glede na pretekle interakcije, te ocene pa se lahko delijo naprej kot priporočila.
Model uporablja pogojno tranzitivnost zaupanja – entiteta A lahko zaupa entiteti C, če A zaupa B kot priporočitelju in lahko ovrednoti njegovo oceno.

Pri tem se uporablja zvezna vrednost zaupanja $tv_T$ med $0$ in $1$, izračunana iz vrednosti priporočil po poti $A \rightarrow B \rightarrow C \rightarrow D$.
Za končno vrednost se uporabi zmnožek delnih zaupanj na poti:
\[
  tv_T = \prod_{i=1}^{n} \frac{rtv(i)}{4} \times tv(T)
\]
kjer $rtv(i)$ predstavlja stopnjo zaupanja v posameznega posrednika, $tv(T)$ pa končno vrednost zaupanja v ciljno entiteto.
Če obstaja več poti med akterjema, se končna vrednost določi kot povprečje rezultatov posameznih poti.

\subsection{Porazdeljeno ocenjevanje zaupanja}

V omrežjih enakovrednih akterjev se zaupanje ocenjuje na decentraliziran način. Vsako vozlišče vodi lastno evidenco interakcij in vrednoti druge glede na uspešnost sodelovanja. Xiong in Liu predlagata metriko, ki temelji na razmerju med pozitivnimi in negativnimi interakcijami:
\[
  T(u,t) = \frac{\sum_{v \in P, v \neq u} S(u,v,t) \, Cr(v,t)}{\sum_{v \in P, v \neq u} I(u,v,t)}
\]
kjer $S(u,v,t)$ označuje zadovoljstvo uporabnika $u$ z $v$ do časa $t$, $Cr(v,t)$ je korekcijski faktor za filtriranje povratnih informacij, $I(u,v,t)$ pa število interakcij med $u$ in $v$.
Rezultat $T(u,t)$ je vedno v intervalu $(0,1)$ in predstavlja trenutno stopnjo zaupanja v entiteto.
Sistem lahko določi pragove $C_1$ in $C_2$, pri čemer entiteta velja za zaupanja vredno, če velja $I(u,t) > C_1$ in $T(u,t) > C_2$.

Takšni sistemi so učinkoviti pri akumuliranju ocen, vendar občutljivi za napačne ali zlonamerne povratne informacije.
Zato številne rešitve uvajajo uteževanje priporočil glede na zanesljivost virov ali časovno starost podatkov.

\subsection{Dinamično posodabljanje zaupanja}

Ker se vedenje akterjev sčasoma spreminja, se mora zaupanje prilagajati novim podatkom.
Posodobitev vrednosti se pogosto izvede z amortizacijskimi ali diskontnimi faktorji, ki dajejo večjo težo nedavnim interakcijam.
Primer takega mehanizma je izračun nove vrednosti zaupanja po zaključeni interakciji:
\[
  T_{\text{new}} = \frac{r + N \, T_{\text{old}} \, e^{-\beta \Delta t}}{1 + N \, e^{-\beta \Delta t}}
\]
kjer $T_{\text{old}}$ predstavlja prejšnjo vrednost zaupanja, $r$ novo oceno po transakciji, $N$ število izvedenih interakcij, $\beta$ faktor dušenja, $\Delta t$ pa čas od zadnje posodobitve.
Na ta način sistem zagotavlja, da nove izkušnje hitreje vplivajo na oceno, stare pa postopoma izgubijo težo.

\subsection{Povezava z našo rešitvijo}
Predstavljeni modeli zaupanja predstavljajo osnovo za delovanje razvitega sistema. Vsak akter v dobavni verigi lahko uporabi svoj način izračuna zaupanja, ki temelji na dokazih, priporočilih ali dinamičnih metrikah. Sistem ne vsiljuje enotnega algoritma, temveč omogoča, da akter sam oblikuje pravila na podlagi podatkov iz ontologije. Rezultati teh izračunov se shranjujejo v verigo blokov, ki deluje kot skupen in trajen zapis ocen. S tem smo združili različne pristope v enoten sistem, kjer je zaupanje merljivo in preverljivo.

\section{Farmacevtske dobavne verige}

\subsection{Značilnosti in pomen sledljivosti}

Farmacevtska dobavna veriga je ena najbolj reguliranih in kompleksnih industrijskih verig. Vključuje proizvajalce, distributerje, prevoznike, lekarne, regulatorne agencije in končne uporabnike \cite{panda2024drug}. Zaradi velikega števila vmesnih akterjev in globalnega obsega poslovanja je nadzor nad izvorom in kakovostjo zdravil zahtevna naloga.

Sledenje zdravilom omogoča identifikacijo posamezne serije skozi celoten življenjski cikel izdelka, od proizvodnje do izdaje pacientu \cite{kayhan2022ensuring}. Natančno beleženje gibanja zdravil preprečuje vstop ponaredkov v sistem in omogoča hitro ukrepanje ob odpoklicih ali neskladjih. Sledljivost je zato neposredno povezana z varnostjo pacientov in zaupnostjo zdravstvenega sistema \cite{world2017study}.

Za zagotavljanje sledljivosti so regulatorji uvedli stroge predpise. Evropska unija je sprejela Direktivo o ponarejenih zdravilih (2011/62/EU) \cite{eu-62-2011} in Delegirano uredbo (EU) 2016/161 \cite{eu-161-2016}, ki zahtevata enolično označevanje embalaže z dvodimenzionalno kodo in uporabo evropskega sistema EMVS (European Medicines Verification System). V ZDA podobno vlogo opravlja zakon DSCSA, ki uvaja popolnoma digitalno sledenje zdravil do leta 2025 \cite{us2013dscsa}.

Ti ukrepi predstavljajo pomembne korake k večji sledljivosti, vendar še vedno temeljijo na centraliziranih bazah podatkov, kjer posamezni akterji nimajo popolnega vpogleda v celotno verigo.

\subsection{Izzivi in tveganja}

Kljub regulacijam dobavne verige zdravil ostajajo izpostavljene številnim tveganjem. Največji problem predstavljajo ponarejena in neustrezna zdravila, katerih število v zadnjih letih narašča. Po podatkih Svetovne zdravstvene organizacije je bilo med letoma 2017 in 2021 zabeleženih 877 incidentov ponarejenih ali neustreznih medicinskih izdelkov, kar pomeni povprečno letno rast za 36 \% \cite{world2024global}. Ti izdelki so bili zaznani v vseh regijah sveta, tudi v državah z razvitimi regulativnimi sistemi. Ponarejena zdravila so lahko neučinkovita ali nevarna, njihova prisotnost pa neposredno ogroža zdravje pacientov in zmanjšuje zaupanje javnosti v zdravstvene sisteme.

Drugo tveganje izhaja iz razdrobljenosti informacijskih sistemov. Podatki o serijah, skladiščih in pošiljkah se pogosto hranijo v ločenih bazah, ki med seboj niso povezane \cite{uddin2021blockchain}. To povzroča zamude, napake in oteženo preverjanje izvora zdravila. Centralizirane rešitve, kot je EMVS, sicer izboljšajo preverjanje avtentičnosti, vendar ne rešujejo težave zaupanja med različnimi organizacijami, ki podatke ustvarjajo in posredujejo.

Poleg ponarejanja in pomanjkljive sledljivosti se pojavljajo tudi tveganja, povezana z motnjami v dobavi zdravil. V kriznih obdobjih, kot so pandemije ali politične napetosti, se zaradi pomanjkanja podatkov o zalogah in zanesljivosti partnerjev pogosto pojavijo nepričakovane prekinitve dobav. Takšni dogodki razkrivajo, kako pomembno je zaupanje med organizacijami in pravočasna izmenjava podatkov.

Kljub tehnološkim in pravnim izboljšavam obstoječi modeli še vedno ne zagotavljajo celovitega vpogleda v dogajanje znotraj verige. Potrebne so rešitve, ki združujejo varno upravljanje podatkov, decentralizirano preverjanje in semantično usklajevanje informacij med akterji.

\subsection{Regulativni okvir in trenutne rešitve}

Evropska zakonodaja temelji na Direktivi o ponarejenih zdravilih (2011\-/62\-/EU) \cite{eu-62-2011} ter Delegirani uredbi (EU) 2016/161 \cite{eu-161-2016}. Skupaj uvajata obvezne varnostne elemente na embalaži, dvodimenzionalno kodo ter sistem EMVS. EMVS omogoča preverjanje serijske številke ob izdaji zdravila, sistemi NMVS pa sodelujejo prek skupnega evropskega vozlišča. Sistem zaznava poskuse ponarejanja v fazi distribucije, ne omogoča pa vpogleda v celotno zgodovino serije, saj temelji na centralizirani arhitekturi \cite{kayhan2022ensuring}.

V ZDA sledljivost ureja zakon DSCSA (Drug Supply Chain Security Act), ki je del zakona DQSA iz leta 2013 \cite{us2013dscsa}. DSCSA uvaja elektronsko izmenjavo treh vrst podatkov: transakcijske informacije, zgodovine transakcij in izjave o skladnosti. Sistem zahteva interoperabilnost in sledljivost do leta 2025. Kljub napredku ameriški model ostaja pretežno centraliziran, podatki pa se prenašajo predvsem neposredno med poslovnimi partnerji \cite{uddin2021blockchain}.

Na mednarodni ravni Svetovna zdravstvena organizacija (WHO) pripravlja smernice za dobre distribucijske prakse in spremlja pojav ponarejenih in neustreznih izdelkov.

Za izpolnjevanje regulativnih zahtev podjetja uporabljajo različne informacijske platforme, med katerimi so najpogostejše centralizirane rešitve, kot so že omenjeni EMVS, TraceLink in SAP ATTP. Te rešitve izboljšujejo sledljivost serij in omogočajo avtomatizirano poročanje regulatorjem, vendar ne omogočajo skupnega vpogleda v podatke ali preverjanja zanesljivosti akterjev po celotni verigi \cite{kayhan2022ensuring}. Podatkovni tokovi ostajajo razdrobljeni, zanašajo pa se tudi na zaupanja vredne posrednike.

V zadnjem desetletju se pojavljajo projekti, ki preučujejo uporabo tehnologije veriženja blokov v farmacevtskih dobavnih verigah. Projekta MediLedger v ZDA in PharmaLedger v Evropi uporabljata porazdeljene zapise za izboljšanje integritete in sledljivosti podatkov \cite{uddin2021blockchain}. Takšni pristopi dokazujejo, da veriga blokov omogoča nespremenljivost zapisov in enostavnejše preverjanje informacij, vendar večinoma rešujejo le problem integritete podatkov.

Kljub temu večina rešitev ostaja usmerjena v zaupanje v podatke, medtem ko je upravljanje zaupanja med posameznimi organizacijami pogosto implicitno, vezano na statične pravice dostopa ali vnaprej dogovorjene pogodbeno–pravne odnose. Manj je pristopov, kjer bi lahko vsak akter izrecno definiral svoje kriterije zaupanja in iz njih izpeljal odločitve o sodelovanju. Prav to vrzel bo zapolnil sistem, predstavljen v tej nalogi, saj kombinira semantični model dobavne verige, subjektivne politike zaupanja in zapis rezultatov v verigo blokov.

\section{Tehnologija veriženja blokov v dobavnih verigah}
Tehnologija veriženja blokov je bila prvič predstavljena leta 2008 v sklopu kriptovalute Bitcoin \cite{nakamoto2008bitcoin}. Osnovna ideja je zasnova podatkovne strukture, ki povezuje zapise v verigo blokov. Vsak blok vsebuje podatke, kriptografski povzetek prejšnjega bloka in časovno oznako. Takšna vezava med bloki preprečuje naknadne spremembe podatkov, saj bi sprememba v enem bloku povzročila neujemanje celotne verige \cite{crosby2016blockchain}.

Veriga blokov deluje v porazdeljenem omrežju, kjer vsako vozlišče hrani kopijo celotne verige. Soglasje o pravilnem zaporedju blokov se doseže s konsenznim mehanizmom. V javnih verigah se pogosto uporabljata t. i. Proof of Work ali Proof of Stake, medtem ko zasebna omrežja uporabljajo lažje mehanizme, ki temeljijo na znanih udeležencih. Zaradi porazdeljene narave sistem ne potrebuje centralne avtoritete, kar omogoča zanesljivo beleženje podatkov v okolju z različnimi deležniki.

Poleg javnih omrežij (kot je Ethereum), so v praksi pogosta zasebna omrežja, kjer je dostop omejen na pooblaščene organizacije. Primer takšnega sistema je Hyperledger Fabric \cite{androulaki2018hyperledger}. Ta pristop omogoča višjo prepustnost, večjo zasebnost podatkov in nadzor nad identitetami v omrežju. V dobavnih verigah so takšne lastnosti ključne, saj morajo podjetja varovati poslovne podatke, hkrati pa zagotavljati zanesljivo izmenjavo informacij.

Verige blokov omogočajo nespremenljiv zapis dogodkov, preverljivo zgodovino podatkov in enoten pogled na transakcije. To so pomembne zahteve farmacevtskih dobavnih verig, saj želimo omogočiti transparenten pogled na transakcije v takšnem sistemu.

\subsection{Pametne pogodbe}
Pametne pogodbe predstavljajo enega ključnih konceptov sodobnih verig blokov. Gre za programe, ki se shranijo in izvajajo neposredno v verigi blokov ter se sprožijo, ko so izpolnjeni določeni pogoji. Ethereum je prvi uvedel splošno platformo za izvajanje pametnih pogodb prek navideznega stroja EVM (Ethereum Virtual Machine), ki omogoča deterministično izvajanje kode na vseh vozliščih omrežja \cite{wood2014ethereum}.

Pametne pogodbe omogočajo avtomatizacijo pravil in procesov, pri čemer ni potrebno zaupati enemu samemu posredniku. Njihovo delovanje je predvidljivo, saj se vedno izvršijo natančno tako, kot je določeno v kodi. Zaradi teh lastnosti se pogosto uporabljajo v scenarijih, kjer je treba uveljaviti dogovorjena pravila med neodvisnimi akterji. V dobavnih verigah se uporabljajo za potrjevanje dogodkov, registracijo premikov blaga, preverjanje skladnosti podatkov, itn.

V modelih, ki temeljijo na verigi blokov, pametne pogodbe služijo tudi kot enotna plast poslovne logike. S tem omogočajo, da vsi akterji delujejo na osnovi istega sklopa pravil. Zlasti je to pomembno v okoljih, kjer se morajo organizacije dogovarjati o interpretaciji podatkov ali izvajanju procesov. Pametne pogodbe z nespremenljivo kodo zagotovijo, da se pravila ne spreminjajo brez soglasja deležnikov.

V predstavljenem sistemu ima pametna pogodba osrednjo vlogo, saj hrani rezultate ocenjevanja zaupanja in omogoča preverjanje vzpostavljenih razmerij med akterji. Pogodba vsebuje funkcije za zapisovanje rezultatov, njihovo posodabljanje in preverjanje stanja zaupanja. Tako tvori register zaupanja, ki je dostopen vsem udeležencem verige.

\subsection{Oraklji v pametnih pogodbah}
Pametne pogodbe lahko zanesljivo izvajajo le pravila nad podatki, ki so že zapisani na verigi blokov.
Ker so verige blokov zasnovane kot zaprt, determinističen sistem, nimajo neposrednega dostopa do zunanjega sveta.
Da bi pametne pogodbe lahko reagirale na dogodke iz realnega okolja (cene na trgu, vremenske razmere, meritve senzorjev, podatke informacijskih sistemov), potrebujemo posrednika, ki te podatke pridobi in jih v preverljivi obliki posreduje na verigo.
Tak posrednik se imenuje orakelj \cite{caldarelli2020oracleproblem}.
V nadaljevanju naloge izraz \emph{orakelj} uporabljamo v tem tehničnem pomenu za neodvisno komponento izven verige, ki iz zunanjih podatkov izračuna oceno zaupanja in pripravi kriptografsko podpisano poročilo, ki se lahko uporabi v pametni pogodbi.

Caldarelli oraklje opiše kot celoten ekosistem, ki povezuje zunanje vire podatkov z decentralizirano aplikacijo \cite{caldarelli2022overview}.
Tipično ga sestavljajo trije deli: vir podatkov, komunikacijski kanal in pametna pogodba.
Vir podatkov je lahko spletni vmesnik API, podatkovna baza, senzor v internetu stvari ali celo človek, ki potrdi nek dogodek.

Komunikacijski kanal predstavlja vozlišče oziroma niz vozlišč, ki podatke zbrane iz vira prevedejo v obliko transakcije in jih pošljejo pametni pogodbi.
Pametna pogodba na drugi strani vsebuje pravila, s katerimi sprejme ali zavrne posredovane podatke in na njihovi podlagi sproži ustrezno logiko.

Ker oraklji ponovno uvajajo zaupanje v zunanje subjekte, se je v literaturi uveljavil pojem \emph{problem oraklja} (angl. oracle problem) \cite{caldarelli2020oracleproblem}.
Veriga blokov sama po sebi nudi lastnosti, kot so nespremenljivost, deterministično izvajanje in odpornost proti cenzuri, vendar te lastnosti ne veljajo avtomatično za podatke, ki prihajajo preko oraklja.
Če vir podatkov ni zanesljiv ali je komunikacijski kanal centraliziran, lahko napačni podatki povzročijo napačno izvedbo pametne pogodbe, ne da bi to bilo na verigi neposredno razvidno.
Zato je v skupnosti poudarjena potreba po jasnem modelu zaupanja za vsak orakelj. Ta mora opisati, kakšen je motiv oraklja, da deluje pošteno, in kako sistem zazna ali kaznuje odstopanja \cite{caldarelli2020oracleproblem}.

V praksi se pojavlja več arhitektur, ki omogočajo delovanje orakljev.
Najpreprostejši so centralizirani oraklji, kjer ena organizacija zbira podatke in jih pošilja pametnim pogodbam.
Tak pristop je enostaven za implementacijo, vendar ponovno uvaja enotno točko zaupanja in napake.
Sledijo jim porazdeljeni oziroma decentralizirani oraklji, kjer več neodvisnih vozlišč pridobiva in agregira podatke ter tako zmanjšuje tveganje manipulacije.
Primer takšne zasnove je sistem ASTRAEA, ki uporabi glasovalno igro z vložki (angl. stake) in ločene vloge udeležencev (oddajatelji trditev, glasovalci, certificiranci), da z ekonomsko motivacijo spodbudi udeležence k poštenemu poročanju o resničnosti trditev \cite{adler2018astraea}.
Podobno pristopajo tudi druge rešitve, kjer se omrežja orakljev, kot je Chainlink, uporabljajo za zbiranje podatkov iz več API-jev in senzorjev ter za izračun dodatnih metrik (npr. reputacije) še pred zapisom v verigo blokov \cite{kochovski2019trust}.

V okviru te naloge oraklje obravnavamo kot ločeno plast med semantičnim delom sistema in verigo blokov.
Podobno kot pri Chainlinku smo zasnovali decentralizirano omrežje pametnih orakljev, kjer teče koda. Omrežje orakljev iz ontologije in pravil zaupanja prebere podatke, izvede izračun zaupanja ter rezultat posreduje pametni pogodbi.
S tem sledimo pristopu, ki ga Kochovski in sodelavci imenujejo \emph{pametni oraklji brez zaupanja} (angl. trustless smart oracles), kjer so oraklji zasnovani tako, da so preverljivi in zamenljivi, veriga blokov pa služi kot nespremenljiv dnevnik rezultatov, ki jih izračunajo oraklji \cite{kochovski2019trust}.
Na ta način odpravimo potrebo po centraliziranem zaupnem posredniku, hkrati pa omogočimo, da se različne implementacije orakljev oz. različni izračuni zaupanja enotno zapišejo v register zaupanja na verigi blokov. Podrobnejši opis delovanja decentraliziranega sistema orakljev si bomo pogledali v poglavju \ref{cht:implementation}.

\subsection{Decentralizirane identitete in preverljiva dokazila}

Decentralizirane identitete (DID) predstavljajo mehanizem za dosledno in preverljivo identifikacijo akterjev v decentraliziranih okoljih. Standard W3C DID opisuje identifikatorje, ki niso vezani na centraliziranega ponudnika, temveč so pod nadzorom lastnika identitete \cite{didcore2020}. DID je globalno enoličen identifikator, na primer \texttt{did:ethr:0x1234...}, pri čemer se podatki o tem, kako identiteto preveriti, hranijo v tako imenovanem DID dokumentu. Ta dokument vsebuje javne ključe, metode preverjanja in dodatne atribute, ki jih sistem lahko uporabi pri avtentikaciji.

Preverljiva dokazila (angl. Verifiable Credentials - VC) so standardizirana digitalna dokazila, ki jih neka entiteta izda drugi entiteti. Model W3C sledi arhitekturi izdajatelj–imetnik–preveritelj. Izdajatelj (Issuer) ustvari in podpiše dokazilo, imetnik (Holder) ga hrani, preveritelj (Verifier) pa ga validira s pomočjo kriptografskih podpisov \cite{vcdata2019}. VC omogočajo prenos lastnosti ali certifikatov na način, ki je preverljiv, varen in ne potrebuje zaupanja v posrednika. V literaturi so prepoznani kot jedrni gradnik digitalnih identitet, ki združujejo suverenost uporabnika, interoperabilnost ter preverljivost podatkov \cite{mazzocca2025survey}.

V predstavljenem sistemu ima vsak akter v ontologiji lasten DID. DID služi kot osnovni identifikator, na katerega se vežejo atributi in podatki v ontologiji. Na ta način sistem ohrani enotno identiteto akterjev med vsemi plastmi arhitekture.

Preverljiva dokazila uporabljamo za opis lastnosti, ki so pomembne za ocenjevanje zaupanja. Regulator, kot je EMA, lahko na primer izda VC, ki potrjuje veljavnost licence, certifikat GMP ali rezultat presoje kakovosti. Ta dokazila so digitalno podpisana in povezana z DID akterja. Orakelj lahko med izračunom zaupanja prebere VC, preveri podpis izdajatelja ter iz podatkov izlušči metrike, ki jih akter uporablja v svoji politiki zaupanja. Če akter v svojih pravilih določi, da mora imeti partner veljavno licenco ali ustrezno oceno presoje kakovosti, lahko te kriterije pridobi neposredno iz VC.

\section{Semantični splet in ontologije}
Semantični splet predstavlja razširitev svetovnega spleta, kjer podatki niso opisani le sintaktično, temveč tudi semantično. To pomeni, da so podatki strukturirani na način, ki omogoča strojno razumevanje in logično sklepanje. Temeljna ideja je, da lahko različni sistemi izmenjujejo podatke v enotnih formatih ter iz njih izpeljujejo nova dejstva, ne da bi se morali dogovoriti o skupni implementaciji ali aplikaciji \cite{lassila2001semantic}.

RDF (Resource Description Framework) je osnovni podatkovni model semantičnega spleta. Podatki so predstavljeni v obliki trojk {subjekt-\allowbreak predikat–\allowbreak objekt}, kar omogoča enostavno združevanje in razširjanje podatkovnih grafov \cite{w3crdfconcepts}. RDF ne predpisuje sheme, zato se lahko prilagodi različnim domenam. Zapis se pogosto izvaja v obliki Turtle, RDF/XML ali JSON-LD.

Nad RDF je zasnovan OWL (Web Ontology Language), ki omogoča formalno opisovanje razredov, lastnosti in odnosov med pojmi. OWL temelji na opisni logiki, kar omogoča avtomatsko sklepanje in preverjanje konsistence ontologije \cite{world2012owl}. Z uporabo razumevalnikov (angl. reasoner) lahko sistem samodejno razvršča primerke v ustrezne razrede, zazna protislovja in izpelje nova dejstva, ki niso bila eksplicitno zapisana.

Ontologije so osrednji gradnik semantičnega spleta. Predstavljajo formalne modele znanja, kjer so pojmi, lastnosti in hierarhije jasno definirani. Tak pristop je posebej uporaben v kompleksnih domenah, kjer nastopa veliko različnih akterjev in množica podatkovnih razmerij. Ontologije omogočajo enotno semantično podlago, ki ni vezana na implementacijo posameznega sistema, temveč na konceptualni model podatkov.

V predstavljenem sistemu ontologija služi kot centralni model znanja za vse akterje farmacevtske dobavne verige. RDF omogoča predstavitev akterjev in njihovih lastnosti kot povezav v grafu, OWL pa dodaja formalne razrede in omejitve, ki omogočajo avtomatsko sklepanje o lastnostih ali vlogah akterjev. Reasoner iz RDF–OWL grafa izlušči podatke, ki jih orakelj nato uporabi pri izračunu zaupanja. Z uporabo semantičnih tehnologij je mogoče podatke strukturirati na način, ki je razširljiv, interoperabilen in neodvisen od posamezne aplikacije. Zato je plast ontologije primerna kot osnova za izmenjavo informacij med različnimi organizacijami in se dobro povezuje s preostalo arhitekturo sistema.

\section{Analiza vrzeli in prispevek te naloge}
V tem poglavju smo predstavili ključne gradnike, ki so pomembni za razumevanje razvitega sistema. Najprej smo opredelili pojem zaupanja in razložili, da gre za večdimenzionalen koncept, ki ga ni mogoče zajeti z eno samo definicijo. Različni modeli poudarjajo kompetentnost, integriteto, dobrohotnost in zanesljivost, v digitalnem okolju pa se temu pridružijo še vidiki varnosti, verodostojnosti in skladnosti z zahtevami sistema. Nato smo opisali modele upravljanja zaupanja v razpršenih sistemih ter pokazali, da se ti modeli večinoma osredotočajo na izračun številčne vrednosti zaupanja na podlagi dokazov, priporočil ali zgodovine interakcij.

V nadaljevanju smo predstavili posebnosti farmacevtskih dobavnih verig. Te verige so močno regulirane in vključujejo veliko različnih akterjev. Regulativa, kot sta evropska direktiva o ponarejenih zdravilih in zakon DSCSA v ZDA, uvaja sledljivost serij in elektronsko izmenjavo podatkov. Kljub temu ostajajo pomembne vrzeli. Obstoječi sistemi večinoma temeljijo na centraliziranih bazah podatkov, kjer posamezna podjetja nimajo celovitega vpogleda v verigo. Ponarejena ali neustrezna zdravila se še vedno pojavljajo, podatkovni tokovi pa so razdrobljeni in vezani na zaupanja vredne posrednike. Trenutne rešitve tako izboljšujejo sledljivost, ne rešujejo pa dovolj dobro vprašanja, komu lahko posamezna organizacija zaupa pri sodelovanju.

Tehnologija veriženja blokov in pametne pogodbe ponujajo nov način upravljanja podatkov v takih okoljih. Verige blokov omogočajo nespremenljiv zapis dogodkov in enoten pogled na podatke v porazdeljenem omrežju. Pametne pogodbe omogočajo izvajanje skupnih pravil, ki veljajo za vse udeležence. Oraklji rešujejo problem povezave med verigo blokov in zunanjimi viri podatkov, vendar ponovno odpirajo vprašanje, komu zaupati pri posredovanju podatkov. Semantični splet in ontologije pa omogočajo formalno predstavitev znanja o akterjih in njihovih lastnostih ter avtomatsko sklepanje nad podatki. V literaturi se ti pristopi le redko združijo v enotno arhitekturo, kjer bi imeli akterji možnost, da na podlagi skupne ontologije definirajo svoje kriterije zaupanja, rezultat teh ocen pa se nato trajno zapiše v verigo blokov.

Na tej točki se pokaže osrednja vrzel. Obstoječe rešitve za sledljivost zdravil se osredotočajo na varnost in integriteto podatkov, upravljanje zaupanja med organizacijami pa je običajno implicitno. Temelji na statičnih pravicah dostopa, pogodbah in ročnih pregledih partnerjev. Po drugi strani številni modeli zaupanja ne upoštevajo posebnosti farmacevtske domene in niso povezani z verigo blokov. Manj je pristopov, kjer bi lahko vsak akter izrecno zapisal svojo politiko zaupanja, jo vezal na skupni semantični model ter rezultate ocenjevanja delil na način, ki ga lahko drugi akterji neodvisno preverijo.

V tabeli~\ref{tab:primerjava-pristopov} povzamemo ključne kategorije pristopov, ki se pojavljajo v literaturi in industrijskih rešitvah. Primerjava poudari razlike med centraliziranimi metodami, reputacijskimi sistemi, semantičnimi modeli in rešitvami na osnovi verige blokov. V zadnjem stolpcu je prikazano, kako sistem, predstavljen v tem delu, združuje elemente teh pristopov in uvaja nove mehanizme, kot so več neodvisnih orakljev, kriptografsko podpisana poročila ter kombinacija dokaznih in telemetrijskih kriterijev.

\begin{landscape}
\begin{table}[ht]
\centering
\caption{Primerjava obstoječih pristopov k upravljanju zaupanja in novosti razvitega prototipa.}
\label{tab:primerjava-pristopov}
\small
\renewcommand{\arraystretch}{1.25}
\begin{tabularx}{\linewidth}{|X|X|X|X|X|}
\hline
\textbf{Pristop} &
\textbf{Vir zaupanja} &
\textbf{Dinamika posodabljanja} &
\textbf{Razlaga odločitve} &
\textbf{Omejitve} \\
\hline

Centralizirani sistemi kakovosti (GDP/GMP) &
ročne revizije, certifikati, agencije &
nizka; periodične revizije &
delna; opis revizij &
centralizacija, počasen odziv, ni kriptografske sledljivosti \\
\hline

Reputacijski modeli (ocene partnerjev) &
mnenja, zgodovina interakcij &
srednja; po vsaki interakciji &
nizka–srednja; pogosto neformalna &
možnost manipulacij, subjektivnost, neprenosljivost ocen \\
\hline

Semantični modeli zaupanja &
ontologije, formalna pravila, sklepanje &
srednja; ob posodobitvi virov &
visoka; formalna razlaga &
ni mehanizma za integriteto na verigi  \\
\hline

Rešitve z uporabo verige blokov (blockchain) &
transakcije, digitalna potrdila &
visoka; dogodki v realnem času &
srednja; podatki preverljivi, razlaga ne &
ne rešujejo izračuna zaupanja; brez večnivojskih kriterijev \\
\hline

\textbf{Predstavljen sistem} &
VC, ontologija, DON &
visoka; vsak izračun sproži nove ocene &
zelo visoka; deterministični + verjetnostni del + zastavice &
odprta vprašanja upravljanja identitet, cena implementacije \\
\hline
\end{tabularx}
\end{table}

  
\end{landscape}


Prispevek te naloge je predlog sistema, ki omenjene elemente poveže v enoten okvir. Ontologija farmacevtske dobavne verige predstavlja skupno semantično plast, v kateri so akterji in njihove lastnosti opisani na strukturiran in razširljiv način. Vsak akter lahko nad temi podatki definira lastna pravila zaupanja in tako določi, katere metrike so zanj pomembne. Mehanizem za sklepanje iz ontologije prebere lastnosti akterjev, uporabi lokalna pravila in za vsak par udeležencev izračuna rezultat zaupanja. Decentralizirani oraklji nato te rezultate prenesejo v pametno pogodbo, ki na verigi blokov hrani register relacij zaupanja. Sistem povezuje tudi decentralizirane identitete (DID) in preverljiva dokazila VC, kadar akterji želijo izvor podatkov podpreti z digitalnimi certifikati.

Tak pristop neposredno naslavlja vrzeli, opisane v tem poglavju. Namesto enotne, vnaprej določene metrike zaupanja omogoča, da vsak akter oblikuje lastno politiko na osnovi skupnega modela podatkov. Rezultati teh različnih politik so kljub temu zbrani v enoten register, ki ga lahko delijo vsi udeleženci verige. S tem sistem podpira različne poglede na zanesljivost partnerjev, hkrati pa ohranja konsistentno predstavitev identitet in podatkovnih lastnosti.

Čeprav je sistem razvit na primeru farmacevtske dobavne verige, je zasnova splošna. Ontologijo in nabor atributov bi bilo mogoče prilagoditi drugim domenam, na primer pametnim mestom ali omrežjem interneta stvari, kjer številni akterji izmenjujejo podatke in storitve. Enak vzorec se ponovi tudi tam: skupni semantični model entitet, subjektivne politike zaupanja, modul za sklepanje in register rezultatov na verigi blokov.

V nadaljevanju naloge bomo opisali, kako je tak sistem konkretno implementiran in kako se odziva v izbranih scenarijih uporabe.


\chapter{Implementacija sistema}
\label{cht:implementation}


V tem poglavju preidemo od konceptualnega modela k dejanski implementaciji prototipa. Na začetku opišemo tehnologije, ki smo jih uporabili. V nadaljevanju bolj podrobno orišemo arhitekturo ter njene ključne komponente. Na koncu predstavimo delovanje sistema skozi tokove podatkov in razložimo, kako so različni deli povezani v celoto.  

Na sliki~\ref{fig:architecture-concept} je prikazan konceptualni prikaz ciljne arhitekture sistema. Diagram ponazarja glavne gradnike: akterje dobavne verige, uporabniški vmesnik, plast znanja (ontologija in shramba preverljivih poverilnic), decentralizirano omrežje orakljev (DON) ter verigo blokov s pametno pogodbo, ki služi kot skupni register zaupanja. Prototip, ki ga opišemo v nadaljevanju, realizira to zasnovo v testnem okolju in omogoča izvajanje ponovljivih scenarijev.

\begin{figure}[H]
  \centering
  \includegraphics[width=0.7\textwidth]{figures/trust_management_architecture_concept.png}
  \caption{Konceptualni prikaz arhitekture sistema za upravljanje zaupanja}
  \label{fig:architecture-concept}
\end{figure}

\section{Uporabljene tehnologije}

Pri razvoju sistema smo uporabili različna orodja in platforme, ki skupaj omogočajo obdelavo semantičnih podatkov, izvajanje izračuna zaupanja ter zapis rezultatov v verigo blokov. V tej sekciji podajamo pregled najpomembnejših tehnologij in njihovih vlog v sistemu.

\paragraph{Apache Jena Fuseki}
Semantični podatki o akterjih in njihovih lastnostih so shranjeni v strežniku Apache Jena Fuseki, ki predstavlja RDF podatkovno bazo z dostopom prek SPARQL vmesnika. Fuseki omogoča centralizirano hrambo ontologije in primerkov ter enoten način pridobivanja podatkov za vse oraklje. Zaradi stabilnosti, enostavne uporabe in dobre združljivosti s standardi RDF je bila izbira naravna. Sistem teče v Docker vsebniku, kar omogoča hitro postavitev in ponovljivost okolja.

\paragraph{Ontologija v OWL/RDF}
Ontologija farmacevtske dobavne verige je zapisana v jeziku OWL, ki definira razrede, lastnosti in odnose med akterji. Podatki so organizirani v RDF trojkah, kar omogoča enostavno obdelavo s knjižnico \texttt{rdflib} \cite{rdflib} v programskem jeziku Python. Na ta način lahko sklepanje temelji na aktualnih podatkih, zapisanih v ontologiji, brez potrebe po relacijski bazi ali ročnem povezovanju.

\paragraph{Ethereum, Solidity in pametna pogodba}
Za zapis rezultatov zaupanja smo uporabili verigo blokov Ethereum.
Pametna pogodba \texttt{TrustGraph}, napisana v jeziku Solidity, hrani relacije zaupanja med akterji.
Ethereum je bil izbran zaradi široke podpore za orodja, stabilnega ekosistema in zmožnosti izvajanja pametnih pogodb brez zaupanja v enega posrednika.
Razvoj in testiranje sta izvedena s pomočjo orodja Foundry, ki nudi lokalno testno omrežje, prevajalnik ter hitre skripte za izvajanje transakcij.

Pametna pogodba omogoča zapis rezultatov izračuna, preverjanje stanja zaupanja in beleženje dogodkov. Deluje kot nespremenljiv register, ki je dostopen vsem akterjem sistema.

\paragraph{Decentralizirane identitete in preverljiva dokazila}
Vsak akter v ontologiji ima dodeljen decentraliziran identifikator (DID).
DID služi kot enoličen identifikator, ki povezuje podatke v ontologiji s predstavnikom akterja v decentraliziranem okolju.
Poleg tega sistem podpira uporabo preverljivih dokazil (VC), ki jih lahko izdajajo regulatorji, kot je EMA.
Ta dokazila vsebujejo metrike, kot so veljavnost licence, GMP skladnost ali rezultati presoje kakovosti.
Orakelj lahko, kadar akter to določi v pravilih, med izračunom zaupanja prebere VC in uporabi podatke kot dodatne kriterije.

\paragraph{Python in osnovne knjižnice}
Del logike sistema je implementiran v programskem jeziku Python, kjer tečejo oraklji, agregator ter orodja za upravljanje ontologije. Python je bil izbran zaradi že razvitih uporabnih knjižnic za obdelavo podatkov, delo z RDF in integracijo z verigami blokov. Spodaj so povzete ključne knjižnice, ki jih sistem uporablja:

\begin{itemize}

\item{\texttt{web3} in \texttt{eth\_account}} -- knjižnica \texttt{web3} omogoča povezavo z omrežjem Ethereum oziroma lokalnim vozliščem Anvil, klicanje funkcij pametne pogodbe in pošiljanje transakcij. Modul \texttt{eth\_account} se uporablja za podpisovanje poročil orakljev in preverjanje podpisov, ki jih agregator prejme.

\item{\texttt{aiohttp}} zagotavlja asinhroni spletni strežnik in odjemalca, ki ju uporabljata agregator in oraklji. Prek njega oraklji prejemajo zahteve za izračun zaupanja, berejo podatke iz Fusekija in pošiljajo poročila nazaj agregatorju.

\item{\texttt{rdflib}} omogoča delo z RDF grafi, razčlenjevanje OWL datotek in manipulacijo ontologije.

\item{\texttt{SPARQLWrapper}} omogoča pošiljanje SPARQL poizvedb strežniku Apache Jena Fuseki in s tem branje lastnosti akterjev iz ontologije

\item{\texttt{requests}} se uporablja pri pomožnih sinhronih HTTP klicih, kjer asinhrona izvedba ni potrebna ali bi otežila implementacijo.

\item{\texttt{click}} se uporablja pri implementaciji ukaznih orodij, kot je \texttt{trustkb}, ki omogoča delo z ontologijo, uvoz podatkov in izvajanje SPARQL poizvedb iz ukazne vrstice.

\item{\texttt{pytest}} knjižnica je bila uporabljena za enotsko testiranje ključnih modulov sistema, saj omogoča enostavno pisanje testov ter avtomatizirano preverjanje logike orakljev in agregatorja.

\item{\texttt{didkit}} -- knjižnica se uporablja v primerih, kjer se digitalna dokazila (VC) izdaja ali preverja neposredno iz Python okolja. Omogoča delo z DID metodami in verifikacijo podpisov, skladno s standardi W3C DID in VC.

\end{itemize}

\paragraph{Pametni oraklji}
Oraklji predstavljajo most med semantično plastjo in verigo blokov.
Vsak orakelj zažene Python modul, ki prebere ontologijo, uporabi lokalna pravila akterja in izračuna vrednosti zaupanja.
Nato preko \texttt{web3.py} pošlje transakcijo pametni pogodbi.
Oraklji so zasnovani modularno, zato jih lahko organizacije poganjajo samostojno, kar omogoča decentralizirano izvajanje izračuna.

\paragraph{Docker}
Vse komponente sistema so zajete v vsebnikih Docker. To nam je omogočilo hitro postavitev razvojnega okolja, enostavno upravljanje odvisnosti in ponovljivost testov.
Vsak orakelj, strežnik Fuseki, in lokalna Ethereum veriga tečejo v ločenih vsebnikih.
Docker je uporabljen izključno za potrebe testiranja med razvojem.


\section{Arhitektura sistema}
\label{sec:arhitektura-sistema}
Arhitektura razvitega sistema temelji na večplastnem pristopu, ki ločuje podatkovni model, mehanizme izračuna in zapis rezultatov zaupanja. Takšna zasnova omogoča jasne meje med posameznimi odgovornostmi, kar olajša razvoj, testiranje in morebitno razširjanje sistema na druge domene. V okviru magistrske naloge smo zasnovali arhitekturo, ki je dovolj splošna, da jo je mogoče uporabiti v različnih razpršenih okoljih, od farmacevtskih dobavnih verig do pametnih mest.

Osnovna ideja arhitekture je bila, da združimo tri ključne tehnologije: semantični splet, pametne oraklje in verigo blokov. Semantični splet omogoča opis akterjev in njihovih lastnosti v ontologiji. Oraklji predstavljajo računsko plast, kjer posamezni akter izvede oceno zaupanja po svojih pravilih. Veriga blokov pa služi kot zanesljiva in nespremenljiva plast, kjer se hranijo rezultati teh ocen.

Sistem se razlikuje od ostalih rešitev, ki se osredotočajo zgolj na sledljivost ali upravljanje podatkov. Omogoča, da vsak akter izrecno definira svoje kriterije zaupanja in jih izvede nad skupnim modelom znanja. Rezultati teh izračunov se zapišejo v decentraliziran register, ki je enak za vse udeležence.

V nadaljevanju poglavja predstavimo zasnovo arhitekture in posamezne plasti, prikažemo njihov medsebojni tok podatkov ter utemeljimo vlogo uporabljenih tehnologij.

\subsection{Pregled večplastne zasnove}

Arhitektura sistema temelji na treh jasno ločenih plasteh, ki skupaj tvorijo celoten potek od zajema podatkov do zapisa rezultatov zaupanja na verigo blokov. Takšna razdelitev omogoča boljše razumevanje posameznih vlog in enostavnejše nadgrajevanje sistema, saj lahko vsako plast obravnavamo neodvisno od drugih.

Na sliki~\ref{fig:context-diagram} je prikazana arhitektura celotnega sistema. Diagram združuje tri glavne sklope: akterje iz dobavne verige na levi, infrastrukturo orakljev in podatkovnih virov na desni ter verigo blokov v sredini. Akterji, kot so regulator, proizvajalec zdravil, prevoznik in lekarna, komunicirajo izključno s pametno pogodbo \texttt{TrustGraph.sol}. Ta beleži rezultate zaupanja in upravlja zahteve za ponovno oceno posameznih entitet.

Ko regulator ali drug akter sproži zahtevo za izračun zaupanja, pametna pogodba pošlje zahtevo omrežju orakljev. Vsak orakelj nato izvede svoj del naloge: iz ontologije in baze TrustKB pridobi podatke, po potrebi preveri pripadajoča preverljiva dokazila, nato pa izračuna rezultat zaupanja. Oraklji rezultate podpišejo in posredujejo agregatorju, ki pripravi končni konsenzni izračun in ga zapiše nazaj v pametno pogodbo.

V skrajnem desnem delu diagrama je prikazana plast znanja. Ta vključuje ontologijo (TrustKB), kjer so zbrani koncepti, njihove lastnosti in meritve, ter skladišče dokazil VC, ki zajema licence, certifikate in druge verodostojne podatke o akterjih. Oraklji pri izračunu združujejo oba vira: semantične lastnosti iz ontologije in verodostojne podatke iz dokazil.

Sistem tako deluje skozi jasno razmejene sklope, kar je razvidno iz celotnega toka na diagramu – od zahteve, preko izračuna, do zapisa končnega rezultata zaupanja na verigi blokov.

\begin{landscape}
  \begin{figure}
    \centering
    \includegraphics[width=\linewidth]{figures/architecture-context.png}
    \caption{Kontekstni pogled sistema za upravljanje zaupanja.}
    \label{fig:context-diagram}
  \end{figure}
\end{landscape}

V nadaljevanju so podrobneje opisane tri plasti, ki sestavljajo arhitekturo:

\begin{enumerate}
  \item \textbf{Plast znanja (Knowledge Layer)}
        V tej plasti se nahaja ontologija dobavne verige in opisne lastnosti akterjev. Ontologija predstavlja skupni referenčni model, ki omogoča enotno interpretacijo podatkov o akterjih. Poleg ontologije se tukaj nahaja tudi skladišče preverljivih dokazil (VC), ki dodaja podatke o licencah, certifikatih in presojah kakovosti.

  \item \textbf{Plast izračuna zaupanja (Trust Resolution Layer)}
        Oraklji v tej plasti izvajajo izračune zaupanja. Podatke pridobijo iz ontologije in preverljivih dokazil ter jih povežejo s pravili, ki jih določi posamezen akter. Pravila so zapisana v JSON obliki in opisujejo pogoje, ki jih mora ocenjovana entiteta izpolnjevati. Oraklji izvedejo izračun, rezultat podpišejo in ga posredujejo agregatorju.

  \item \textbf{Plast registra zaupanja na verigi blokov (Blockchain Trust Registry Layer)}
        Ta plast vključuje pametno pogodbo, ki hrani rezultate ocenjevanja zaupanja. Pogodba deluje kot decentraliziran register, ki omogoča vpogled v trenutna in pretekla razmerja zaupanja med akterji. Vanjo se vpisujejo tako posamezni odgovori orakljev kot konsenzni rezultat evalvacije.
\end{enumerate}

V nadaljevanju so opisane tehnične podrobnosti posamezne plasti ter njihove medsebojne povezave.

\subsubsection{Plast znanja}

Plast znanja predstavlja osnovni vir informacij, na katerem poteka izračun zaupanja. Vsebuje semantični model dobavne verige zdravil, podatke o posameznih akterjih ter dodatna preverljiva dokazila, ki opisujejo regulatorne lastnosti ali skladnost z zahtevami. V tej plasti se združujejo podatki iz dveh virov: ontologije, ki opisuje akterje in njihove lastnosti, ter preverljivih dokazil (VC), ki dopolnjujejo semantični model z digitalno podpisanimi trditvami. Oraklji te podatke uporabljajo pri izračunu zaupanja v nadaljnjih plasteh sistema.


\paragraph{Ontologija (OWL/RDF)}

Ontologija farmacevtske dobavne verige je napisana v jeziku OWL in zapisana v formatu RDF. Vsebuje razrede, ki predstavljajo posamezne tipe akterjev, kot so \textit{Proizvajalec}, \textit{Prevoznik}, \textit{Lekarna} in \textit{Regulator}. Tem razredom so pripisane lastnosti, ki opisujejo njihovo vedenje ali skladnost s standardi, na primer:

\begin{itemize}
  \item \texttt{hasDeliveryPunctuality} – delež pravočasnih dostav,
  \item \texttt{hasTempViolationRate} – delež temperaturnih odstopanj pri prevozu,
  \item \texttt{hasLicense} – veljavnost poslovne licence,
  \item \texttt{hasGMP} – skladnost s standardi GMP \cite{nally2016good},
  \item \texttt{hasAuditScore} – rezultat presoje kakovosti.
\end{itemize}

Ontologija vsebuje tudi posamezne primerke akterjev, na primer \textit{Pfizer} ali \textit{DHL}, ki imajo pripisane konkretne vrednosti lastnosti. Spodaj je prikazan primer zapisa prevoznika v RDF/Turtle obliki:

\begin{lstlisting}
ex:DHL a ex:Transporter ;
  ex:hasDeliveryPunctuality "0.94"^^xsd:decimal ;
  ex:hasTempViolationRate "0.02"^^xsd:decimal ;
  ex:hasLicense "true"^^xsd:boolean .
\end{lstlisting}

Ontologija je shranjena na strežniku Apache Jena Fuseki, ki omogoča izvajanje poizvedb tipa SPARQL 1.1 preko spletnega vmesnika. Oraklji tako dostopajo do podatkov z neposrednimi poizvedbami, kot je na primer naslednja poizvedba za pridobitev lastnosti prevoznika DHL:

\begin{minipage}{\linewidth}
\begin{lstlisting}
PREFIX ex: <http://example.org/trust#>

SELECT ?punctuality ?tempRate ?license
WHERE {
  ex:DHL ex:hasDeliveryPunctuality ?punctuality .
  ex:DHL ex:hasTempViolationRate ?tempRate .
  ex:DHL ex:hasLicense ?license .
}
\end{lstlisting}
\end{minipage}

Ker je ontologija zgrajena v OWL, omogoča tudi uporabo razumevalnika (angl. reasoner). Razumevalnik lahko samodejno razvrsti akterje v razrede (npr. \texttt{TrustedTransporter}), zazna nedoslednosti in izpelje lastnosti, ki niso zapisane neposredno. Vsak orakelj lahko razumevalnik po potrebi zažene lokalno in s tem pridobi izpeljane podatke.

\newpage
\paragraph{Preverljiva dokazila (VC)}

Poleg podatkov iz ontologije sistem uporablja še preverljiva dokazila (VC), ki jih izdajajo regulatorji ali druge zaupanja vredne organizacije. Ta dokazila vsebujejo trditve o akterjih, kot so veljavnost licence, GMP skladnost ali datum zadnje presoje kakovosti. Dokazila so zapisana v formatu JSON-LD in digitalno podpisana.

Primer dokazila, ki ga izda regulator:

\begin{minipage}
{\linewidth}
\begin{lstlisting}
{
  "id": "vc-gmp-pfizer-001",
  "type": ["VerifiableCredential", "GMPCertificate"],
  "issuer": "did:example:ema",
  "credentialSubject": {
    "id": "did:example:pfizer",
    "hasGMP": true
  },
  "proof": {
    "type": "Ed25519Signature2018",
    "signatureValue": "3f9ac..."
  }
}
\end{lstlisting}
\end{minipage}

Preverljiva dokazila so shranjena v ločeni podatkovni bazi. Ob izračunu zaupanja orakelj preveri kriptografski podpis izdajatelja, iz podatkov izlušči relevantne lastnosti in jih združi z informacijami iz ontologije. Tako sestavi popoln nabor kriterijev, ki jih akter uporabi pri ocenjevanju zaupanja.

Plast znanja združuje semantične opise akterjev in njihove lastnosti z regulatorno potrjenimi podatki iz preverljivih dokazil. Oraklji pri izračunu zaupanja dostopajo do obeh virov: iz Fusekija pridobijo opisne metrike, iz VC-jev pa dodatne podatke, ki jih ontologija ne vsebuje. S tem plast znanja omogoča enoten, strukturen in preverljiv nabor podatkov, na katerem temelji celoten izračun zaupanja.

\subsubsection{Plast izračuna zaupanja}
Plast izračuna zaupanja predstavlja osrednjo procesno komponento arhitekture. Njena naloga je, da podatke iz plasti znanja pretvori v konkretne odločitve o zaupanju in jih posreduje verigi blokov. V tej plasti se nahaja omrežje orakljev in agregator, ki skupaj tvorita neodvisen mehanizem sklepanja. Oraklji izvajajo izračune, agregator pa skrbi za preverjanje rezultatov in zapis končnega stanja v pametno pogodbo.

\paragraph{Oraklji}
Ko sistem prejme zahtevo za izračun zaupanja, pametna pogodba sproži dogodek, ki služi kot signal orakljem, da morajo izvesti novo oceno. Oraklji ne delujejo kot pasivni odjemalci, temveč aktivno spremljajo verigo blokov in zaznavajo zahteve, ki jih je treba obdelati. Za podani subjekt nato pridobijo vse podatke iz plasti znanja: opisne lastnosti iz ontologije ter pripadajoča preverljiva dokazila iz baze poverilnic VC. Na ta način je izračun zaupanja vedno vezan na najnovejše in preverljive podatke.

Pridobljene podatke orakelj združi s pravili ocenjevalca. Pravila določajo, katere lastnosti mora ocenjevan akter izpolnjevati in pod kakšnimi pogoji. Orakelj izvede ocenjevanje in pripravi strukturirano poročilo, ki vsebuje identifikator subjekta, končno odločitev, numerično oceno, kriptografske povzetke uporabljenih dokazil ter čas izračuna. Poročilo je serializirano v JSON obliki in digitalno podpisano z zasebnim ključem oraklja. S tem postane rezultat preverljiv in vezan na točno določen orakelj.

\paragraph{Agregator}
Podpisano poročilo orakelj pošlje agregatorju, ki deluje kot zbirni člen med oraklji in pametno pogodbo. Agregator preveri digitalni podpis, preveri, ali je poročilo prišlo od dovoljenega oraklja, ter ga zapiše na verigo kot posamično oddajo. Ko prejme zadostno število poročil, izvede konsenz nad prejetimi odločitvami in oceno zaupanja ter prek transakcije posreduje končni rezultat v pametno pogodbo \texttt{TrustGraph}.

Na visoki ravni potek izračuna poteka v naslednjih korakih:
\begin{enumerate}
  \item pametna pogodba na verigi blokov prejme zahtevo za nov izračun zaupanja in sproži dogodek,
  \item oraklji poslušajo te dogodke, za podanega akterja preberejo podatke iz ontologije in pripadajočih preverljivih dokazil,
  \item podatke združijo s pravili ocenjevalca in izvedejo izračun zaupanja,
  \item rezultat pretvorijo v strukturirano poročilo, ga digitalno podpišejo in pošljejo agregatorju,
  \item agregator preveri prejeta poročila, izračuna konsenz in prek transakcij zapiše posamezne oddaje ter končni rezultat v pametno pogodbo.
\end{enumerate}

Na ta način plast izračuna zaupanja povezuje semantične podatke z decentraliziranim zapisom rezultatov. Oraklji omogočajo neodvisno izvajanje izračunov, agregator pa poskrbi za preverjanje in enoten zapis v register zaupanja na verigi blokov. Plast tako tvorimo most med modelom znanja in verigo blokov ter predstavlja ključen element celotne arhitekture.

\subsubsection{Plast registra zaupanja na verigi blokov}

Plast registra zaupanja predstavlja končni sloj arhitekture, kjer se rezultati izračuna zaupanja trajno zapišejo. Njena vloga je zagotoviti podatkovno plast, ki ni odvisna od posamezne organizacije in ki omogoča preverljiv, dosleden ter nespremenljiv zapis vseh odločitev o zaupanju. V tej plasti se nahaja pametna pogodba \texttt{TrustGraph}, ki deluje kot decentraliziran register razmerij zaupanja med akterji.

Pametna pogodba je zasnovana tako, da sprejema rezultate evalvacij orakljev in jih zapisuje v podatkovne strukture na verigi. Podpisi orakljev, konsenz agregatorja in mehanizmi preverjanja identitet poskrbijo, da v register pridejo le preverjene in verodostojne informacije. Vsak zapis je povezan z identifikatorjem akterja, časom vnosa in dodatnimi metapodatki, kar omogoča transparenten zgodovinski vpogled v spremembe zaupanja skozi čas.

Ko agregator prejme dovolj veljavnih poročil orakljev, izvede postopek združevanja in prek transakcije pokliče funkcijo v pametni pogodbi, ki zapiše tako posamične oddaje kot končni konsenzni rezultat. Pogodba sproži dogodke, ki služijo kot signal akterjem ali nadrejenim sistemom, da je za določen subjekt na voljo nova ocena zaupanja. Vsaka sprememba je trajno zapisana v verigo blokov, zato je mogoče rezultate neodvisno preveriti tudi za nazaj.

Register zaupanja hrani dve vrsti podatkov. Prva vrsta so posamične oddaje orakljev, ki predstavljajo neobdelane rezultate izračuna. Druga vrsta so konsenzni rezultati, ki jih agregator izračuna na podlagi več oddaj in jih pogodba obravnava kot veljavno končno stanje zaupanja za posameznega akterja. Na ta način lahko sistem ohranja celovito sliko o tem, kako so se rezultati spreminjali in kako je bila sprejeta končna odločitev.

Zapis v register omogoča tudi poizvedovanje nad trenutnim ali zgodovinskim stanjem zaupanja. Akterji v dobavni verigi lahko kadar koli preverijo, ali je določen partner trenutno ocenjen kot zaupanja vreden ali ne. Ker podatki niso shranjeni v centraliziranem sistemu, temveč v porazdeljeni verigi blokov, se vnosov ne da naknadno spreminjati ali brisati. To daje sistemu lastnost dolgoročne preverljivosti, kar je ključno v okoljih, kjer se sprejemajo odločitve z regulatornimi posledicami.

Plast registra zaupanja tako predstavlja stičišče celotnega sistema: prejme rezultate iz računske plasti, jih trajno zapiše in omogoča zanesljiv vpogled v relacije zaupanja v vsakem trenutku. V nadaljevanju bomo opisali funkcije pametne pogodbe \texttt{TrustGraph}, podatkovne strukture, dogodke ter način, kako zunanji akterji dostopajo do registra.

\section{Tok podatkov v sistemu}

\subsection{Celoten potek izračuna zaupanja}

Na sliki~\ref{fig:seq-diagram-recalculation} je prikazan celoten proces izračuna zaupanja, ki vključuje vse komponente sistema: aplikacijo, pametno pogodbo \texttt{TrustGraph.sol}, decentralizirano omrežje orakljev, strežnik Fuseki ter bazo preverljivih dokazil VC. V nadaljevanju se osredotočimo na funkcije uporabljene v procesu. Pri vsakem koraku so navedene ključne funkcije.

\begin{landscape}
  \begin{figure}
    \centering
    \includegraphics[width=1.44\textwidth]{figures/seq-diagram-recalculation.png}
    \caption{Diagram poteka ob izračunu zaupanja za subjekt.}
    \label{fig:seq-diagram-recalculation}
  \end{figure}
\end{landscape}

\begin{enumerate}
  \item \textbf{Aplikacija ali uporabnik ustvari zahtevo}\newline
    Uporabnik preko aplikacije ali klica API sproži postopek za izračun zaupanja za želenega akterja. Akterjev DID pred klicem funkcije pretvorimo v binarno vrednost \texttt{bytes32} s pomočjo zgoščevalne funkcije Keccak-256, ki je značilna za Ethereumov ekosistem. Aplikacija preko odjemalca RPC pokliče funkcijo na pametni pogodbi:
    \begin{quote}\texttt{requestTrustReport(subject, ttl)}\end{quote}

    \item \textbf{Pametna pogodba obdela zahtevo in izda signal za izračun}\newline
    Po klicu funkcije \texttt{requestTrustReport(subject, ttl)} pametna pogodba preveri, ali je za izbranega akterja že aktivna zahteva ter ali ima klicatelj ustrezna dovoljenja (če je seznam dovoljenih nastavljen). Ko so pogoji izpolnjeni, pogodba ustvari nov identifikator zahteve \texttt{requestId}.  
    Identifikator temelji na zgoščeni kombinaciji DID akterja, časovnega žiga, naslova pošiljatelja in zaporedne številke zahteve. Pogodba shrani rok za oddajo (\texttt{deadline}) in nato sproži dogodek:
    \begin{quote}\texttt{TrustOracleRequested(requestId, subject)}\end{quote}
    Ta dogodek služi kot edini mehanizem obveščanja decentraliziranega omrežja orakljev, da je treba izvesti nov izračun zaupanja. Obenem funkcija klicatelju tudi vrne \texttt{requestId}, na podlagi katerega lahko spremlja stanje in kasneje pridobi končni rezultat.

  \item \textbf{Orakelj zazna novo zahtevo in začne proces izračuna}\newline
    Vsak orakelj neprestano posluša za nove dogodke, zapisane v verigi blokov (angl.\ \textit{event logs}). Ko zazna dogodek \texttt{TrustOracleRequested}, preveri, ali mora obdelati podani \texttt{requestId}.  
    Če je orakelj nastavljen za tovrstne izračune, začne postopek ocenjevanja. Najprej pridobi vse potrebne podatke iz plasti znanja: semantične lastnosti akterja preko poizvedb SPARQL strežniku Fuseki ter pripadajoča preverljiva dokazila iz baze VC. Orakelj tako zagotovi, da bo izračun temeljil na najnovejših, verodostojnih in digitalno podpisanih podatkih.
    Šele ko so vsi podatki zbrani in preverjeni, orakelj preide na sam izračun zaupanja, kjer uporabi pravila ocenjevalca.

  \item \textbf{Orakelj osveži podatke v plasti znanja}\newline
    Pred izračunom zaupanja mora orakelj pridobiti najnovejše podatke iz dveh ločenih virov:
    \begin{itemize}
      \item semantične lastnosti akterja iz ontologije prek poizvedb SPARQL na strežniku Fuseki,
      \item veljavna preverljiva dokazila (VC), ki jih orakelj prebere iz lokalne baze dokazil.
    \end{itemize}
    Pridobivanje podatkov iz ontologije poteka prek SPARQL končne točke, ki vrne lastnosti subjekta (npr.\ točnost dostav, licenčne statuse, GMP skladnost). Preverljiva dokazila se preverijo z validacijo digitalnega podpisa izdaje. S tem orakelj zagotovi, da izračun temelji na najnovejših in verodostojnih podatkih.

  \item \textbf{Izvedba izračuna glede na pravila ocenjevalca}\newline
    Ko ima orakelj na voljo vse relevantne podatke, izvede izračun zaupanja. Pri tem uporabi pravila, ki si jih bomo v nadaljevanju pogledali na testnih scenarijih. Rezultat tega koraka je numerična ocena in binarna odločitev, ali akter izpolnjuje zahtevane pogoje. Poleg tega orakelj pripravi tudi metapodatke, kot so:
    \begin{itemize}
      \item čas izračuna (\texttt{asOf}),
      \item zgoščena vrednost pravil (\texttt{policyHash}),
      \item zgoščena vrednost uporabljenega VC (\texttt{credentialHash}).
    \end{itemize}

  \item \textbf{Orakelj ustvari poročilo in ga podpiše}\newline
    Orakelj pripravi končno poročilo v JSON obliki, ki vsebuje ključne informacije o izračunu zaupanja: identifikator subjekta, odločitev, oceno, zastavice in pripadajoče povzetke. Preden ga pošlje agregatorju, mora orakelj poročilo kriptografsko podpisati. Podpis se izvede z uporabo metode \texttt{sign\_message}, ki je del knjižnice web3.py. Ta temelji na Ethereumovi shemi EIP-191. Najprej se pripravi deterministična kanonična oblika JSON zapisa, ta pa se nato podpiše z zasebnim ključem oraklja. Podpis omogoča agregatorju, da preveri, ali je poročilo prišlo od zaupanja vrednega oraklja.

  \item \textbf{Agregator sprejme poročila in izvede preverjanje}\newline
    Agregator deluje kot zbirno vozlišče, ki prejema poročila posameznih orakljev prek HTTP vmesnika. Pred prejemom poročila izvede naslednja preverjanja:
    \begin{itemize}
      \item ali podpis pripada dovoljenemu oraklju,
      \item ali poročilo ni podvojeno,
      \item ali se zgoščene vrednosti pravil in dokazil ujemajo z lokalnimi podatki.
    \end{itemize}
    Vsako veljavno poročilo agregator takoj zapiše v pametno pogodbo z uporabo funkcije:
    \begin{quote}\texttt{recordOracleSubmission(requestId, evaluatorHash,\break report, credentialHash)}\end{quote}
    Na ta način je posamezna ocena oraklja shranjena na verigi blokov, neodvisno od končnega konsenza.

  \item \textbf{Agregator izvede konsenz in zapiše končni rezultat}\newline
    Ko agregator prejme dovolj veljavnih poročil (dosežen kvorum), izvede konsenz nad rezultati. Pri tem kombinira odločitev, oceno in dodatne zastavice, nato pa pripravi končno poročilo. Končni rezultat se zapiše v pametno pogodbo z uporabo funkcije:
    \begin{quote}\texttt{fulfillTrustReport(requestId, report)}\end{quote}
    Pametna pogodba preveri, ali je agregator pooblaščen, ali je \texttt{requestId} veljaven in ali zahteva ni pretekla. Če so vsi pogoji izpolnjeni, pogodba zapiše konsenzni rezultat v verigo blokov  in sproži dogodek \texttt{TrustOracleFulfilled}.
    Pogodba preveri identiteto klica, ujemanje ID in iztek roka, preden zapiše končni rezultat v:
    \begin{quote} \texttt{subjectMetrics[subject]} \end{quote}
    in sprožen je dogodek \texttt{TrustOracleFulfilled}, ki zaključi postopek.
\end{enumerate}

Ker je celoten postopek sestavljen iz več medsebojno odvisnih korakov, se v naslednjih podpoglavjih osredotočimo na bolj podroben pregled posameznih delov. Najprej pogledamo notranje delovanje omrežja orakljev in agregacije rezultatov, nato pa še na branje podatkov iz registra zaupanja.
\newpage
\subsection{Agregiranje rezultatov v omrežju orakljev}

Na sliki~\ref{fig:seq-diagram-aggregation} je prikazan podroben potek obdelave poročil znotraj decentraliziranega omrežja orakljev (DON). Diagram prikazuje fazo po tem, ko oraklji prejmejo zahtevo za izračun zaupanja, in se osredotoča na izvajanje izračuna, podpisovanje poročil ter postopek agregacije v agregatorju. Ta proces je ključen za zagotavljanje verodostojnih rezultatov, saj združi poročila več neodvisnih orakljev v en sam, konsenzni izračun.

\begin{figure}[htbp]
  \centering
  \includegraphics[width=0.95\textwidth]{figures/seq-diagram-aggregation.png}
  \caption{Potek agregiranja rezultatov v decentraliziranem omrežju orakljev}
  \label{fig:seq-diagram-aggregation}
\end{figure}

V nadaljevanju so opisani vsi ključni koraki, prikazani v diagramu.

\begin{enumerate}

  \item \textbf{Zajem podatkov iz plasti znanja}\newline
    Orakelj pred izvedbo izračuna pridobi aktualne podatke iz dveh virov:
    \begin{itemize}
      \item podatki ontologije prek strežnika Fuseki (poizvedbe SPARQL), ki vključujejo opisne metrike akterja,
      \item preverljiva dokazila (VC), ki jih orakelj preveri glede veljavnosti in morebitnega preklica.
    \end{itemize}
    Ti podatki predstavljajo vhodne metrike, ki se uporabijo pri izračunu zaupanja.

  \item \textbf{Izvedba lokalnega izračuna zaupanja}\newline
    Orakelj izvede hibridni algoritem ocenjevanja, ki združuje preverjanje pravil ter statistične metrike.  
    V tej fazi orakelj pripravi tudi oba ključna povzetka:
    \begin{itemize}
      \item \texttt{policy\_hash}, ki predstavlja zgoščeno vrednost pravil akterja,
      \item \texttt{credential\_hash}, ki označuje VC dokazilo uporabljeno oceni.
    \end{itemize}

  \item \textbf{Pošiljanje podpisanega poročila agregatorju}\newline
    Orakelj oblikuje strukturirano poročilo, ga serializira v kanonični JSON zapis in digitalno podpiše z metodo \texttt{sign\_message} (EIP-191).  
    Podpisano poročilo nato pošlje agregatorju prek zahteve HTTP (\texttt{POST\break /reports}), skupaj z identifikatorjem zahteve.
    Da agregator poročilo sprejme, mora orakelj poslati podatke v standardizirani obliki \texttt{Signed\-Oracle\-Report}.  
    Ta vsebuje dva dela:
    \begin{itemize}
      \item \texttt{OracleReport} – dejanski rezultat izračuna,
      \item \texttt{node\_id} in \texttt{signature} – identiteto oraklja in kriptografski podpis poročila.
    \end{itemize}

    Struktura \texttt{OracleReport} mora vsebovati naslednja polja:
    \begin{itemize}
      \item \texttt{subject} – 32-bajtni identifikator subjekta (zgoščena vrednost DID),
      \item \texttt{decision} – končna odločitev (bool),
      \item \texttt{score} – številčna ocena zaupanja v odstotkih,
      \item \texttt{flags} – bitna opozorilna polja,
      \item \texttt{as\_of} – čas izračuna.
    \end{itemize}

    Pred podpisom se poročilo pretvori v kanonično JSON obliko s stabilnim urejanjem ključev.  
    Podpis se izvede z metodo \texttt{sign\_message} (EIP-191) iz knjižnice \texttt{web3.py}.  
    Agregator poročilo sprejme le, če je podpis veljaven, naslov \texttt{node\_id} pripada dovoljenemu oraklju, poročilo ni podvojeno in \texttt{policy\_hash} ter \texttt{credential\_hash} se ujemata z lokalnimi podatki agregatorja.
    Standardizirana struktura omogoča, da lahko vsako poročilo neodvisno preverimo in da se rezultati več orakljev združijo v konsenzni izračun, ki se zapiše v pametno pogodbo.

  \item \textbf{Preverjanje podpisov in veljavnosti poročil}\newline
    Agregator ob prejemu poročila preveri:
    \begin{itemize}
      \item ali podpis pripada dovoljenemu oraklju,
      \item ali je poročilo unikatno glede na kombinacijo \texttt{request\_id} in \texttt{node\_id},
      \item ali se zgoščene vrednosti pravil in dokazil ujemajo z njegovimi lokalnimi podatki.
    \end{itemize}
    Veljavna poročila agregator takoj zapiše v pametno pogodbo s funkcijo \texttt{recordOracleSubmission}.

  \item \textbf{Doseganje kvoruma poročil}\newline
    Agregator hrani prejeta poročila in spremlja, ali je dosežen kvorum (minimalno število poročil, potrebnih za izvedbo konsenza).
    Če kvorum ni dosežen, agregator čaka in nadaljuje zbiranje poročil dokler le ta ni dosežen ali do izteka časa čakanja (\texttt{ttl}).

  \item \textbf{Agregacija rezultatov znotraj DON}\newline
    Ko je kvorum dosežen, agregator izvede agregacijo. Poleg odločitve in ocene agregator združi tudi \texttt{flags}, ki predstavljajo bitna opozorilna polja. Vsak bit ima posebej določen pomen, ki označuje posebne okoliščine ali tveganja, ki so se pojavila med izračunom:
    \begin{itemize}
      \item {FLAG\_NO\_DATA (bit 0)} – orakelj ni našel nobenih vhodnih podatkov za subjekt (npr. subjekt v ontologiji ne obstaja ali nima nobenih lastnosti).
      \item {FLAG\_LOW\_SCORE (bit 1)} – verjetnostna ocena zaupanja je padla pod minimalni prag.
      \item {FLAG\_DISAGREEMENT (bit 2)} – različni deli orakljevega notranjega izračuna so podali neskladne odločitve.
      \item {FLAG\_VC\_REVOKED (bit 3)} – preverljivo dokazilo (VC), uporabljeno pri izračunu, je označeno kot preklicano.
    \end{itemize}

    Ti biti se lahko nastavijo v orakljih funkcijah za izračun. Agregator pri združevanju uporabi \emph{bitno večino}: vsak bit v končnem rezultatu je nastavljen, če ga ima vsaj polovica oddanih poročil. Zastavice tako predstavljajo povzetek opozoril, ki jih je zaznalo celotno omrežje orakljev.


  \item \textbf{Zapis konsenznega rezultata v pametno pogodbo}\newline
    Končni agregirani rezultat se v verigo blokov zapiše s funkcijo:
    \begin{quote}\texttt{fulfillTrustReport(requestId, report)}\end{quote}
    Pametna pogodba preveri identiteto klicatelja, veljavnost identifikatorja zahteve in rok izvedbe.  
    Ob uspešnem zapisu sproži dogodek \texttt{TrustOracleFulfilled}, ki drugim komponentam signalizira, da je nova ocena na voljo.

\end{enumerate}

Agregator povezuje izračune več orakljev v enoten rezultat, pametna pogodba pa poskrbi, da je končni izračun trajno in preverljivo zapisan v verigi blokov. V naslednjem podpoglavju si bomo ogledali, kako akterji v sistemu pridobijo te rezultate in kako poteka branje podatkov iz registra zaupanja.

\newpage
\subsection{Branje rezultatov zaupanja}
Po zaključenem izračunu in zapisu rezultatov v pametno pogodbo lahko akterji iz sistema pridobijo podatke o zaupanju na dva načina: (i) kot posamezne oddaje orakljev in (ii) kot agregiran, konsenzni rezultat omrežja orakljev (DON). 

\subsubsection{Branje posamičnih izračunov orakljev}

Pametna pogodba \texttt{TrustGraph.sol} hrani posamične rezultate vsakega oraklja v podatkovni strukturi \texttt{oracleSubmissions\-[evaluator]\-[subject]}. Ti podatki predstavljajo neagregirane rezultate, ki jih je posamezen orakelj izračunal za dani subjekt.
Slika~\ref{fig:seq-diagram-read-single} prikazuje potek branja posamičnih oddaj.

\begin{figure}[H]
\centering
\includegraphics[width=0.92\textwidth]{figures/seq-diagram-read-single.png}
\caption{Branje posamične oddaje oraklja iz registra zaupanja}
\label{fig:seq-diagram-read-single}
\end{figure}

Branje posamičnih oddaj je koristno za različne namene. Če je potrebno preveriti, kako je določen orakelj ocenil subjekt, ali pa analizirati morebitna neskladja med oraklji, lahko akterji neposredno dostopajo do teh podatkov. Do teh podatkov dostopamo z direktnim klicem funkcije za branje iz javnega slovarja:
\begin{quote}
\texttt{oracleSubmissions(evaluatorHash, subject)} 
\end{quote}

Rezultat vključuje:
\begin{itemize}
  \item binarno odločitev,
  \item numerično oceno,
  \item zastavice (flags),
  \item čas izračuna,
  \item zgoščene vrednosti pravil in dokazil,
  \item ETH naslov oraklja.
\end{itemize}

\subsubsection{Branje agregiranega (konsenznega) rezultata}
Konsenzni rezultat, ki ga zapiše agregator, predstavlja končno in uradno odločitev omrežja orakljev. 

\begin{figure}[H]
\centering
\includegraphics[width=0.92\textwidth]{figures/seq-diagram-read-consensus.png}
\caption{Branje konsenznega rezultata ocenjevanja iz pametne pogodbe}
\label{fig:seq-diagram-read-consensus}
\end{figure}

Ta rezultat je v pametni pogodbi zapisan v strukturi:
\begin{quote}
\texttt{subjectMetrics[subject]}
\end{quote}

Kot vidimo na \ref{fig:seq-diagram-read-consensus}, lahko do rezultatov dostopamo prek javne funkcije:
\begin{quote}
\texttt{getTrustMetrics(subject)}
\end{quote}

Ta funkcija vrne:
\begin{itemize}
  \item \textbf{odločitev} (\texttt{decision}) – končni izračun zaupanja,
  \item \textbf{oceno} (\texttt{score\_bps}) – numerična vrednost v osnovnih točkah,
  \item \textbf{zastavice} (\texttt{flags}) – bitna opozorila, ki so bila prisotna v večini oddaj,
  \item \textbf{čas} (\texttt{asOf}) – čas izračuna konsenza.
\end{itemize}

Konsenzni rezultat je tisti, ki ga uporabi večina akterjev v sistemu, saj predstavlja stabilen in preverjen izračun celotnega omrežja orakljev.

% -----------------------------------------
% 6 Evalvacija sistema in varnostna analiza
% -----------------------------------------

\chapter{Evalvacija}

V tem poglavju predstavimo celovito evalvacijsko analizo razvitega sistema za upravljanje zaupanja. Namen poglavja je prikazati, kako posamezne komponente delujejo skupaj v realističnih scenarijih in kako sistem obnašanju pod vplivom različnih vhodnih podatkov, politik ter sprememb v okolju. Evalvacija je zato zasnovana kot kombinacija eksperimentalne postavitve, podrobnega pregleda algoritmov orakljev in zaporedja scenarijev, ki ponazarjajo ključne procese: izračun zaupanja, agregacijo rezultatov ter odziv sistema na spremembe v podatkih.

Najprej opišemo eksperimentalno okolje in konfiguracijo vseh komponent v vsebnikih Docker, kar zagotavlja ponovljivost in transparentnost izvedbe. Nato predstavimo algoritme izračuna zaupanja, ki jih uporabljajo trije oraklji, zasnovani na različnih konceptualnih modelih: hibridnem semantičnem pristopu, dokaznem modelu, temelječem na preverljivih poverilnicah, ter telemetrijsko usmerjenem pristopu. V osrednjem delu poglavja sledijo podrobno izpeljani scenariji, ki prikazujejo delovanje sistema od začetne zahteve do končnega zapisa na verigo, skupaj z dejanskimi številčnimi rezultati posameznih orakljev ter agregatorja. Poglavje zaključimo z evalvacijo odziva sistema na spremembe vhodnih podatkov in s tem pokažemo, kako se ocena zaupanja dinamično prilagodi novim razmeram. Celotna predstavitev tako omogoča sistematičen vpogled v delovanje rešitve, njeno zanesljivost ter transparentnost postopka izračuna zaupanja.

\section{Predstavitev scenarijev delovanja}
\label{sec:scenariji-delovanja}

\subsection{Eksperimentalno okolje in konfiguracija sistema}
\label{subsec:eksperimentalno-okolje}

Preden predstavimo konkretne scenarije uporabe, najprej opišemo, kako je postavljen eksperimentalni sistem. Vse komponente zaganjamo v orodju \texttt{Docker}, kar nam zagotavlja ponovljivo okolje in jasno ločitev med posameznimi storitvami. S tem zmanjšamo odvisnost od lokalne konfiguracije in omogočimo, da sistem testiramo kakor, da bi bile vse komponente že nameščene v oblaku.

\subsubsection{Vsebniki in omrežje}
Sistem sestavlja več vsebnikov Docker, ki so povezani v skupno virtualno omrežje in delujejo z orodjem \texttt{Docker Compose}. V sekciji \ref{subsec:algoritmi-orakljev} so opisani posamezni vsebniki in načini ocenjevanja zaupanja za vsak orakelj.

\paragraph{Veriga blokov} Prvi vsebnik vsebuje testno Ethereum omrežje s pomočjo orodja Anvil. Na njem namestimo  pametno pogodbo \texttt{TrustGraph.sol}. Na omrežju takoj dobimo tudi 10 prednastavljenih računov z velikimi količinami testnih ETH kovancev, ki jih oraklji in agregator uporabljajo za plačevanje transakcijskih stroškov. Zato prve račune dodelimo agregatorju in orakljem. Potrebujemo tudi en račun za namestitev pametne pogodbe dodelimo le tega regulatorju EMA, kateri bo imel potem tudi pravico do sprožanja novih izračunov zaupanja.

\paragraph{Baza znanja} V vsebniku Fuseki je naložena domenska ontologija. Ta je dostopna preko SPARQL končne točke, ki jo oraklji uporabljajo za pridobivanje semantičnih lastnosti akterjev. V eksperimentalnem okolju uporabljamo ontologijo, ki ga vsi oraklji delijo. Celotna ontologija, uporabljena za scenarije, je na voljo v prilogi \ref{app:trust-ontology}.

\paragraph{Oraklji} Za evalvacijo smo pripravili tri neodvisne oraklje, ki izvajajo svoje politike izračuna zaupanja. Vsak orakelj teče v svojem vsebniku Docker in je konfiguriran z ločeno datoteko, ki določa pravila ocenjevalca, nabor vhodnih podatkov ter težo posameznih kriterijev. Oraklji so implementirani v Pythonu in uporabljajo knjižnice za dostop do verige blokov (web3.py), baze znanja (rdflib, SPARQLWrapper) ter za kriptografske operacije. V sekciji \ref{subsec:algoritmi-orakljev} so podrobneje predstavljeni algoritmi izračuna zaupanja za vsak orakelj.

\paragraph{Agregator} Zbiranje poročil posameznih orakljev teče v vsebniku agregatorja. Ta preverja njihove podpise in objavi končno odločitev v pametni pogodbi na verigi. Tudi agregator je implementiran v Pythonu in uporablja podobne knjižnice kot oraklji.

\paragraph{Odjemalska aplikacija}, ki v našem primeru nastopa kot ukazna vrstica oziroma skripta, s katero sprožimo zahtevo za izračun zaupanja in kasneje preberemo rezultat z verige. V realnem primeru bi bil to lahko npr. spletni vmesnik ali integracija v obstoječi ERP sistem akterja.

Kontejnerji si delijo notranje Docker omrežje, zato med seboj komunicirajo preko domen DNS (\texttt{fuseki:3030}, \texttt{ethereum-node:8545}).

Agregator pri združevanju rezultatov ne pozna podrobnosti posameznih politik. Vsako poročilo obravnava kot podpisano izjavo oraklja, ki vsebuje identifikator subjekta, numerično oceno zaupanja, diskretno odločitev (npr.\ \emph{zaupanja vreden} ali \emph{nezaupanja vreden}) ter morebitne zastavice. Na osnovi konfiguriranega praga in števila prejetih poročil izračuna končno odločitev, ki jo zapiše v pametno pogodbo.

\paragraph{Telemetrijski podatki}
Poleg ontologije in preverljivih poverilnic sistem uporablja tudi agregirane
telemetrijske metrike za izbrane transporterje. V eksperimentalnem okolju so
te metrike zapisane v JSON datoteki v naslednji obliki:

\begin{verbatim}
{
  "http://example.org/trust#DHL": {
    "tempDeviation": 0.02,
    "lateDeliveries": 0.01,
    "alerts": 0
  },
  "http://example.org/trust#MedLogix": {
    "tempDeviation": 0.03,
    "lateDeliveries": 0.02,
    "alerts": 0
  },
  "http://example.org/trust#UPS": {
    "tempDeviation": 0.04,
    "lateDeliveries": 0.03,
    "alerts": 1
  },
  "http://example.org/trust#AcmeTransport": {
    "tempDeviation": 0.08,
    "lateDeliveries": 0.05,
    "alerts": 3
  }
}
\end{verbatim}

V prototipu so vrednosti zapisane statično in služijo kot simulacija. V realnem okolju bi enak format predstavljal projekcijo podatkov iz zunanjih sistemov (npr.\ transportnega informacijskega sistema, IoT senzorjev povezanih na verigo, itn.), ki bi jih oraklji periodično brali prek API-jev ali sporočilnih vodil in na njihovi podlagi sproti posodabljali oceno zaupanja.

\subsection{Algoritmi izračuna zaupanja v orakljih}
\label{subsec:algoritmi-orakljev}

V poglavju~\ref{sec:trust-management-distributed-systems} smo predstavili različne pristope k upravljanju zaupanja v razpršenih sistemih: dokazne modele, priporočilne modele, porazdeljeno ocenjevanje ter dinamično posodabljanje zaupanja. Ti koncepti tvorijo teoretično podlago za zasnovo naših orakljev. V nadaljevanju pokažemo, kako vsak orakelj uporabi drugačen pogled na zaupanje.
\begin{itemize}
    \item \textbf{Orakelj A} uporablja hibridni pristop, ki kombinira dokazne elemente (licence, certifikati) z dinamično metriko, podobno priporočilnim in porazdeljenim modelom.
    \item \textbf{Orakelj B} je izrazito \emph{dokazni} – odloča skoraj izključno na podlagi preverljivih poverilnic, podobno kot hierarhični in PGP-sistemi iz podpoglavja o dokaznih modelih.
    \item \textbf{Orakelj C} je \emph{telemetrijsko usmerjen} in se naslanja na dinamično posodabljanje zaupanja preko operativnih metrik, kar je skladno z modeli, ki večjo težo dajejo nedavnim interakcijam.
\end{itemize}

Čeprav so v prototipu algoritmi implementirani v isti kodni bazi, jih v nadaljevanju obravnavamo kot tri samostojne oraklje, od katerih ima vsak svoj lastni algoritem izračuna zaupanja.

\subsubsection*{Skupna struktura izračuna}

Ne glede na specifičen pogled vsak orakelj za dani subjekt izračuna:
\begin{enumerate}
    \item deterministični del (pravila iz politik),
    \item verjetnostni del (številska verjetnostna ocena),
    \item kombinirano odločitev.
\end{enumerate}

\paragraph{Deterministični del}
Deterministični del temelji na politikah, ki so zapisane v konfiguracijskih datotekah (\texttt{policy.json}). V njih so za posamezne tipe subjektov (proizvajalec, distributer, transporter) definirani kriteriji in pragovi, na primer:
\begin{itemize}
    \item \texttt{hasDeliveryPunctuality} $\geq 0{,}99$,
    \item \texttt{hasTempViolationRate} $\leq 0{,}05$,
    \item \texttt{hasAuditScore} $\geq 0{,}90$,
    \item \texttt{hasGDPVC} $=$ \texttt{true}.
\end{itemize}
Če subjekt za vse zahtevane kriterije zadošča pogojem, deterministični del vrne odločitev \emph{PASS}, sicer \emph{FAIL}. Ta del je najboljši približek \emph{dokaznim modelom zaupanja}, kjer se odločitve sprejemajo na podlagi preverljivih dokazov in jasno določenih pragov.

\paragraph{Verjetnostni del}
Verjetnostni del iste kriterije pretvori v številsko verjetnostno oceno v intervalu $[0,1]$, ki jo orakelj kasneje pretvori v osnovne točke (angl.\ \emph{basis points}, 0--10000). Za vsak kriterij se vodi Beta posterior, ki se posodablja po principu eksponentno uteženega drsečega povprečja (EWMA). S tem se približamo obnašanju modelov iz podpoglavij o priporočilnih modelih in dinamičnem posodabljanju zaupanja.

Pri prvem branju ontologije se kriterij inicializira na osnovi trenutne vrednosti $x$ in normalizirane zaželenosti $s_0 \in [0,1]$ (višja vrednost pomeni bolj zaželen izid). Parametra porazdelitve Beta sta:
\[
    \alpha = 1 + s_0 \cdot \textit{obs\_scale}, \qquad
    \beta = 1 + (1 - s_0) \cdot \textit{obs\_scale},
\]
kjer je v implementaciji uporabljena lestvica $\textit{obs\_scale} = 9$. Za »popolno« izpolnjen pogoj ($s_0 \approx 1$) dobimo pričakovano vrednost porazdelitve beta $\mathbb{E}[X] \approx 0{,}91$, kar predstavlja visoko, a ne absolutno zaupanje.

Pri novih opazovanjih posterior posodobimo z uporabo faktorja EWMA $\gamma \approx 0{,}2$, tako da nove informacije hitreje vplivajo na verjetnostno oceno, starejše pa postopoma izgubljajo težo. To je konceptualno podobno formuli za dinamično posodabljanje zaupanja v podpoglavju o amortizacijskih faktorjih, kjer nove interakcije s časom prevladajo nad starimi.

\paragraph{Kombinirana odločitev}
Za posameznega akterja se verjetnostna ocena izračuna kot uteženo povprečje srednjih vrednosti beta posteriorjev aktivnih kriterijev. Uteži so zapisane v polju \texttt{\_weights} v politiki in se normalizirajo tako, da zadošča:
\[
    \sum_i w_i = 1, \qquad w_i \geq 0.
\]

Vsak orakelj uporablja svoj prag \textit{verjetnostne ocene}, $\textit{prob\_threshold}$, ki ga izražamo na lestvici 0--1 (v implementaciji pa v osnovnih točkah). Kombinirana odločitev ima obliko:
\[
    \textit{combined} = \textit{deterministic} \;\text{OR}\; \left(\textit{prob\_score} \geq \textit{prob\_threshold}\right),
\]
kjer je \textit{deterministic} deterministična odločitev (\emph{PASS}/\emph{FAIL}), \textit{prob\_score} pa izračunana verjetnostna ocena. Na ta način se trda pragovna pravila (dokazni pogled) kombinirajo z mehko, številsko oceno, ki je bližje priporočilnim in porazdeljenim modelom zaupanja.

V nadaljevanju opišemo konkretne konfiguracije in primere za tri oraklje, ki utelešajo različne pristope iz poglavja~\ref{sec:trust-management-distributed-systems}.

\subsubsection*{Orakelj A (hibridni pogled)}

Orakelj A uporablja hibridni pogled, ki združuje regulatorne kriterije, kakovost delovanja in informacije iz preverljivih poverilnic. S tem v praksi kombinira dokazni model (digitalna dokazila, licence) z dinamičnim ocenjevanjem in uteženjem kriterijev preko Beta posteriorjev. Njegov prag za verjetnostno oceno je nastavljen na:
\[
    \textit{prob\_threshold}^{(\text{A})} = 0{,}75.
\]

\paragraph{Kriteriji in uteži}
Za posamezne tipe subjektov uporablja naslednje kriterije in uteži:
\begin{itemize}
    \item \textbf{Transporter:}
    \begin{itemize}
        \item \texttt{hasDeliveryPunctuality} $\geq 0{,}99$ (utež $0{,}7$),
        \item \texttt{hasTempViolationRate} $\leq 0{,}05$ (utež $0{,}3$).
    \end{itemize}
    \item \textbf{Distributor:}
    \begin{itemize}
        \item \texttt{hasLicense} $=$ \texttt{true} (utež $0{,}4$),
        \item \texttt{hasGDPVC} $=$ \texttt{true} (utež $0{,}5$).
    \end{itemize}
    \item \textbf{Manufacturer:}
    \begin{itemize}
        \item \texttt{hasAuditScore} $\geq 0{,}90$ (utež $0{,}6$).
    \end{itemize}
\end{itemize}

\paragraph{Primer: MedLogix (distributer)}
Za distributerja MedLogix v trenutni ontologiji velja \texttt{license=true} in \texttt{GDPVC=true}. Deterministični del zato vrne \emph{PASS}. Oba kriterija imata visoko zaželenost, zato se Beta posteriorja inicializirata s srednjo vrednostjo približno $0{,}91$. Po normalizaciji uteži (približno $0{,}44$ in $0{,}56$) dobimo verjetnostno oceno:
\[
    \textit{prob\_score} \approx 0{,}91 \geq 0{,}75.
\]
Kombinirana odločitev je:
\[
    \textit{combined}^{(\text{A}, \text{MedLogix})} = \textit{PASS}.
\]

\paragraph{Primer: SkyFreight (transporter)}
Za transporterja SkyFreight sta zabeleženi vrednosti \texttt{punctuality} $= 0{,}90$ in \texttt{tempViolationRate} $= 0{,}08$. Deterministični pogoji niso izpolnjeni, ker $0{,}90 < 0{,}99$ in $0{,}08 > 0{,}05$, zato deterministični del vrne \emph{FAIL}. Tudi normalizirana zaželenost $s_0$ za oba kriterija je nizka, kar vodi v nizko srednjo vrednost Beta posteriorja in verjetnostno oceno $\textit{prob\_score} < 0{,}75$. Končna odločitev oraklja je:
\[
    \textit{combined}^{(\text{A}, \text{SkyFreight})} = \textit{FAIL}.
\]

\subsubsection*{Orakelj B (dokazni, VC-centričen pogled)}

Orakelj B je zasnovan tako, da v ospredje postavlja preverljive poverilnice. Konceptualno je najbližje dokaznim modelom zaupanja iz poglavja~\ref{sec:trust-management-distributed-systems}, kjer imata veriga potrdil in kriptografska dokazila ključno vlogo pri inicializaciji zaupanja. Njegov prag za verjetnostno oceno je strožji:
\[
    \textit{prob\_threshold}^{(\text{B})} = 0{,}8.
\]

\paragraph{Kriteriji in uteži.}
Glavne kombinacije kriterijev so:
\begin{itemize}
    \item \textbf{Distributor:}
    \begin{itemize}
        \item \texttt{hasGDPVC} $=$ \texttt{true} (utež $0{,}6$),
        \item \texttt{hasLicense} $=$ \texttt{true} (utež $0{,}3$),
    \end{itemize}
    kar po normalizaciji daje približno $0{,}67$ in $0{,}33$.
    \item \textbf{Manufacturer:}
    \begin{itemize}
        \item \texttt{hasAuditScore} $\geq 0{,}92$ (utež $0{,}4$).
    \end{itemize}
    \item \textbf{Transporter:}
    \begin{itemize}
        \item \texttt{hasRecallRate} $\leq 0{,}02$ (utež $0{,}5$).
    \end{itemize}
\end{itemize}

Ključna značilnost je, da brez ustreznih preverljivih poverilnic odločitev nagiba k \emph{FAIL}, tudi če so druge operativne metrike razmeroma dobre, kar odraža močan poudarek na formalnih dokazih in skladnosti.

\paragraph{Primer: MedLogix (distributer)}
Ker ima MedLogix veljavne poverilnice in licenco, deterministični del vrne \emph{PASS}. Za oba kriterija dobimo srednjo vrednost Beta posteriorja približno $0{,}91$, uteži so po normalizaciji $0{,}67$ in $0{,}33$, zato:
\[
    \textit{prob\_score} \approx 0{,}91 \geq 0{,}8,
\]
in kombinirana odločitev ostane:
\[
    \textit{combined}^{(\text{B}, \text{MedLogix})} = \textit{PASS}.
\]

\paragraph{Primer: HealthChain (transporter)}
Za transporterja HealthChain je v bazi znanja zabeležen \texttt{recallRate} $= 0{,}015$, kar deterministično zadošča pogoju \texttt{hasRecallRate} $\leq 0{,}02$ in daje \emph{PASS}. Pri verjetnostnem delu pa je zaželenost pri vrednosti blizu zgornje meje nizka (npr.\ $s_0 \approx 0{,}25$), kar vodi v srednjo vrednost Beta posteriorja približno $0{,}30$ in:
\[
    \textit{prob\_score} \approx 0{,}30 < 0{,}8.
\]
Kombinirana odločitev je zato:
\[
    \textit{combined}^{(\text{B}, \text{HealthChain})} = \textit{FAIL},
\]
kar ilustrira strožji, dokazni pristop: deterministični \emph{PASS} sam po sebi ni zadosten brez visoke verjetnostne ocene.

\subsubsection*{Orakelj C (telemetrija-prvi pogled)}

Orakelj C uporablja telemetrijo transporta (časovne serije, agregirane metrike) kot primarni vir informacij. Konceptualno se približa porazdeljenim in dinamičnim modelom zaupanja iz poglavja~\ref{sec:trust-management-distributed-systems}, kjer se zaupanje gradi na osnovi kumulativnih izkušenj in se sproti prilagaja novim dogodkom. Njegov prag za verjetnostno oceno je:
\[
    \textit{prob\_threshold}^{(\text{C})} = 0{,}7.
\]

\paragraph{Kriteriji in uteži}
Za proizvajalce in distributerje uporablja naslednje kriterije:
\begin{itemize}
    \item \textbf{Manufacturer:}
    \begin{itemize}
        \item \texttt{hasTempDeviationScore} $\leq 0{,}10$ (utež $0{,}5$),
        \item \texttt{hasGDPVC} $=$ \texttt{true} (utež $0{,}4$).
    \end{itemize}
    \item \textbf{Distributor:}
    \begin{itemize}
        \item \texttt{hasDeliveryPunctuality} $\geq 0{,}95$ (utež $0{,}5$).
    \end{itemize}
\end{itemize}

Pri transporterjih ta orakelj v trenutni konfiguraciji ne izračunava ocene, saj zanje ni definiranih kriterijev; v takem primeru subjekt v tem pogledu ni ocenjen.

\paragraph{Primer: BioPharm (proizvajalec)}
Za proizvajalca BioPharm sta zabeležena \texttt{tempDev} $= 0{,}08$ in \texttt{GDPVC=true}. Oba pogoja sta deterministično izpolnjena (\emph{PASS}), zaželenost je visoka in oba kriterija dobita srednjo vrednost Beta posteriorja približno $0{,}91$. Verjetnostna ocena je zato:
\[
    \textit{prob\_score} \approx 0{,}91 \geq 0{,}7,
\]
kombinirana odločitev pa:
\[
    \textit{combined}^{(\text{C}, \text{BioPharm})} = \textit{PASS}.
\]

\paragraph{Primer: CanaryLabs (proizvajalec)}
Za proizvajalca CanaryLabs sta zabeležena \texttt{tempDev} $= 0{,}12$ in \texttt{GDPVC=false}. Deterministični del zato vrne \emph{FAIL}. Ker sta temperaturna odstopanja prevelika in ni poverilnice o skladnosti, je tudi verjetnostna ocena nizka ($\textit{prob\_score} \ll 0{,}7$). Kombinirana odločitev je:
\[
    \textit{combined}^{(\text{C}, \text{CanaryLabs})} = \textit{FAIL}.
\]

\medskip

Takšna predstavitev treh orakljev neposredno gradi na modelih iz poglavja~\ref{sec:trust-management-distributed-systems}: Orakelj A predstavlja hibridni semantičen in dinamično podprt pristop, Orakelj B predstavlja strogo dokazno-kriptografski pogled, Orakelj C pa dinamično, na meritvah temeljeno ocenjevanje. V scenarijih, ki sledijo, bomo za izbranega subjekta pokazali, kako različni algoritmi privedejo do različnih, a razložljivih odločitev o zaupanju.

\subsection{Agregacija poročil orakljev}
\label{subsec:agregator-scenariji}

V scenarijih, predstavljenih v nadaljevanju, vedno nastopi enak vzorec: za določenega akterja (npr.\ transportno podjetje ali distributerja) pametna pogodba \texttt{TrustGraph.sol} sproži zahtevo po poročilu o zaupanju, oraklji A, B in C izračunajo lastne ocene in jih pošljejo agregatorju, ta pa iz treh poročil izpelje enotno odločitev, ki jo zapiše v verigo blokov.

V opisu scenarijev bomo tabelarično pokazali, kako so posamezne ocene orakljev A, B in C videti pred agregacijo, ter kako je iz njih izpeljana enotna odločitev, ki je na voljo ostalim akterjem v dobavni verigi.

\subsubsection*{Pravila agregacije}

Za dane tri (ali več) veljavnih poročil agregator uporabi preprosta deterministična pravila:

\paragraph{Skupni subjekt in politika}

Najprej preveri, ali so vsa poročila skladna glede na tri kriterije:
\begin{itemize}
  \item vsa poročila veljajo za istega subjekta (isti zgoščeni identifikator),
  \item vsa poročila uporabljajo enak povzetek politike (\texttt{policy\_hash}),
  \item povzetek poverilnic (\texttt{credential\_hash}) ni v nasprotju z shrambo poverilnic (če ta obstaja).
\end{itemize}
Če katero od teh preverjanj pade, agregator poročil ne združi in zahteva v
scenariju velja za neuspešno.

\paragraph{Diskretna odločitev (da/ne)}
Vsak orakelj poda diskretno odločitev, ali je subjekt zaupanja vreden ali ne.
Agregator uporabi preprosto večinsko pravilo:
\[
  \text{odločitev} =
  \begin{cases}
    \text{true}, & \text{če vsaj dva od treh orakljev poročata \emph{true}},\\
    \text{false}, & \text{sicer}.
  \end{cases}
\]
V scenarijih to pomeni, da bo subjekt označen kot zaupanja vreden, če se vsaj dva od treh pogledov (hibridni, dokazni, telemetrijski) strinjata, da so pogoji izpolnjeni.

\paragraph{Številčna ocena zaupanja}
Vsako poročilo vsebuje številčno oceno zaupanja v razponu [0, 10000]. Agregator v trenutni verziji uporablja kar povprečje ocen vseh orakljev.

Povprečje je končna številčna ocena zaupanja, ki se zapiše v pametno
pogodbo in jo v scenarijih prikazujemo kot \emph{agregirano} vrednost.

\paragraph{Zastavice (flags)}
Oraklji z zastavicami označijo posebna stanja, npr.\ pomanjkanje podatkov, nizko oceno ali uporabo preklicanih poverilnic. Agregator za vsak posamezen bit zastavice preveri, ali ga nastavi vsaj polovica orakljev. Če je tako, je ta bit nastavljen tudi v skupnem rezultatu. V scenarijih to pomeni, da se opozorila, na katera opozori večina orakljev, prenesejo tudi v končno oceno, opozorila posameznega “osamelca” pa lahko ostanejo samo v njegovem individualnem poročilu.

\subsubsection*{Zapis agregirane odločitve na verigo}

Ko je agregacija uspešna, agregator za dani \texttt{request\_id} izvede klic \texttt{fulfill\-Trust\-Report} v pametni pogodbi \texttt{TrustGraph.sol}. V
verigo blokov se zapiše:
\begin{itemize}
  \item identifikator subjekta (zgoščena identiteta),
  \item agregirana diskretna odločitev (zaupanja vreden da/ne),
  \item agregirana številčna ocena zaupanja,
  \item agregirane zastavice (večinska opozorila),
  \item časovni žig in povzetek politike.
\end{itemize}

V scenarijih, ki sledijo, bomo zato vedno prikazali dva nivoja:
\begin{enumerate}
  \item \emph{posamična poročila} orakljev A, B in C (njihove odločitve,
        ocene in zastavice),
  \item \emph{agregirano odločitev}, ki jo agregator zapiše na verigo in jo
        ostali akterji uporabljajo kot referenčno oceno zaupanja za danega
        subjekta.
\end{enumerate}
Tako je jasno razvidno, kako se različni algoritmi (hibridni, dokazni,
telemetrijski) zlivajo v eno, soglašano oceno, ki je dosegljiva vsem
udeležencem dobavne verige.

\subsection{Scenarij 1: Ocena zaupanja transporterja DHL}
\label{subsec:scenarij-1}

V prvem scenariju prikažemo celoten potek ocene zaupanja za konkretnega
akterja v dobavni verigi: transportno podjetje \texttt{DHL}, ki skrbi za
prevoz zdravil med proizvajalci, distributerji in lekarnami. Namen scenarija
je pokazati, kako trije različni pogledi (Orakelj A, B in C) vodijo do
deloma različnih rezultatov in kako agregator iz njih izpelje enotno, na
verigi zapisano odločitev.

\subsubsection*{Opis izhodišča}

Za subjekt \texttt{DHL} v trenutni konfiguraciji veljajo naslednja dejstva:
\begin{itemize}
  \item v ontologiji je označen kot \emph{transporter},
  \item za njega obstaja veljavna preverljiva poverilnica skladnosti (VC), kar
        je razvidno iz neničelnega povzetka poverilnic
        (\texttt{credential\_hash}),
  \item poverilnica je shranjena v skupnem repozitoriju VC na strežniku, do
        katerega imajo dostop vsi oraklji,
  \item telemetrijske metrike kažejo visoko pravočasnost dostav in sprejemljiva
        temperaturna odstopanja v hladni verigi.
\end{itemize}

Na tej osnovi:
\begin{itemize}
  \item Orakelj A (hibridni model) kombinira ontologijo, politike in VC ter
        pri \texttt{DHL} zazna nekaj slabših kazalnikov (zato vrne precej
        nižjo številčno oceno kot preostala oraklja, odločitev pa ostane negativna).
  \item Orakelj B (dokazni, VC model) se nasloni skoraj izključno
        na prisotnost veljavne poverilnice (ki je dostopna vsem orakljem) in
        \texttt{DHL} oceni zelo visoko.
  \item Orakelj C (telemetrija) uporablja predvsem telemetrijo in
        kombinacijo kazalnikov točnosti ter temperature in vrne pozitivno
        odločitev z visoko, a nekoliko nižjo oceno kot Orakelj B.
\end{itemize}

\subsubsection*{Potek izvedbe scenarija}

Scenarij se izvede v naslednjih korakih:

\begin{enumerate}
  \item \textbf{Sprožitev zahteve za poročilo.}
        Regulator ali drug pooblaščen akter za transporterja \texttt{DHL}
        sproži zahtevo za poročilo o zaupanju. V prototipu to izvedemo z
        ukazom, ki prek skripte pošlje transakcijo
        \texttt{requestTrustReport} pametni pogodbi \texttt{TrustGraph.sol}.
        Pogodba pri tem ustvari nov identifikator zahteve
        \texttt{request\_id} in izda dogodek \texttt{Trust\-Oracle\-Requested}.
  \item \textbf{Izračun poročil v orakljih.}
        Oraklji A, B in C spremljajo dogodke pametne pogodbe. Ko prejmejo
        dogodek za nov \texttt{request\_id} in subjekt \texttt{DHL}, osvežijo
        ontologijo, naložijo VC iz skupnega repozitorija in telemetrijo ter
        izvedejo svoje algoritme:
        \begin{itemize}
          \item Orakelj A (hibridni) uporabi politike in semantične podatke ter
                za \texttt{DHL} izračuna negativno odločitev z nizko oceno.
          \item Orakelj B (VC-centričen) preveri veljavnost poverilnice in
                subjekt oceni zelo visoko, brez opozoril.
          \item Orakelj C (telemetrija-prvi) na podlagi telemetrijskih metrik
                izračuna pozitivno odločitev in visoko oceno, ki odraža
                dobro operativno delovanje.
        \end{itemize}
        Vsak orakelj pripravi podpisano poročilo \texttt{SignedOracleReport}.
  \item \textbf{Pošiljanje poročil agregatorju.}
        Oraklji pošljejo svoja podpisana poročila agregatorju preko HTTP
        vmesnika \texttt{/reports}. Agregator preveri podpise in identitete
        vozlišč ter poročila začasno shrani pod isti \texttt{request\_id}.
  \item \textbf{Agregacija rezultatov.}
        Ko agregator za dani \texttt{request\_id} prejme poročila vseh treh
        orakljev, uporabi pravila agregacije iz
        podpoglavja~\ref{subsec:agregator-scenariji}: večinsko glasovanje
        za diskretno odločitev, povprečje za številčno oceno ter večinsko
        pravilo za zastavice.
  \item \textbf{Zapis končne odločitve na verigo.}
        Agregator pokliče funkcijo \texttt{fulfillTrustReport} v pametni
        pogodbi in za \texttt{DHL} zapiše končno odločitev in številčno
        oceno zaupanja, ki sta od takrat naprej vidni vsem udeležencem
        sistema.
\end{enumerate}

\subsubsection*{Dejanski rezultati orakljev in agregatorja}

Tabela~\ref{tab:scenarij1-dhl} prikazuje dejanske rezultate poročil orakljev
A, B in C ter agregatorja za subjekt \texttt{DHL}. Vrednosti so izražene v
osnovnih točkah (0--10000).

\begin{table}[ht]
  \centering
  \caption{Rezultati orakljev A, B in C ter agregatorja za scenarij 1 (transporter DHL).}
  \label{tab:scenarij1-dhl}
  \begin{tabularx}{\textwidth}{l X c c c}
    \hline
    Vir & Tip algoritma & Odločitev & Številčna ocena & Zastavice \\
    \hline
    Orakelj A & hibridni      & \emph{FAIL} & 2382 & 0x00 \\
    Orakelj B & dokazni (VC)  & \emph{PASS} & 9500 & 0x00 \\
    Orakelj C & telemetrija   & \emph{PASS} & 8879 & 0x00 \\
    Agregator & agregirano    & \emph{PASS} & 6920 & 0x00 \\
    \hline
  \end{tabularx}
\end{table}

Interpretacija posameznih poročil:

\begin{itemize}
  \item \textbf{Orakelj A} vrne negativno odločitev (\emph{FAIL}) z nizko
        oceno $2382$ in brez zastavic (\texttt{0x00}). Hibridni pogled na
        podlagi politik in semantičnih podatkov subjekt ocenjuje previdno,
        vendar iz same vrednosti ne izpelje dodatnega opozorila.
  \item \textbf{Orakelj B} vrne pozitivno odločitev (\emph{PASS}) z zelo
        visoko oceno $9500$ in brez zastavic (\texttt{0x00}). Ker ima
        \texttt{DHL} veljavno preverljivo poverilnico v skupnem repozitoriju
        VC, VC-centričen algoritem oceni, da so formalni pogoji za zaupanje
        izpolnjeni.
  \item \textbf{Orakelj C} vrne pozitivno odločitev (\emph{PASS}) z visoko
        oceno $8879$ in brez zastavic. Telemetrijske metrike (pravočasne
        dostave, stabilna temperatura) kažejo dobro operativno delovanje, zato
        telemetrijsko usmerjen pogled subjekt oceni kot zaupanja vreden.
\end{itemize}

Agregator iz teh treh poročil izvede:

\begin{itemize}
  \item \textbf{Diskretna odločitev.} Dva od treh orakljev (B in C) poročata
        \emph{PASS}, eden (A) poroča \emph{FAIL}. Po večinskem pravilu je
        končna odločitev \emph{PASS} – subjekt \texttt{DHL} je v agregiranem
        pogledu ocenjen kot zaupanja vreden.
  \item \textbf{Številčna ocena.} Ocene treh orakljev so $2382$, $8879$ in
        $9500$. Aritmetična sredina teh treh vrednosti je približno $6920$,
        zato agregator zapiše končno oceno zaupanja $6920/10000$.
  \item \textbf{Zastavice.} Noben orakelj ne nastavi opozorilnih zastavic,
        zato agregator po večinskem pravilu ohrani vrednost \texttt{0x00}.
        V tem scenariju končni rezultat ne vsebuje globalnih opozoril in
        predstavlja “čist” pozitiven zapis.
  \item \textbf{Politike in poverilnice.} Vsa tri poročila uporabljajo
        enak \texttt{credential\_hash}, kar pomeni, da se sklicujejo na isto
        preverljivo poverilnico iz skupnega repozitorija VC. Agregator pri tem
        preveri, da poverilnica v lokalni shrambi ni označena kot preklicana.
\end{itemize}

Za ostale akterje v dobavni verigi je rezultat scenarija naslednji: ob
poizvedbi nad \texttt{TrustGraph.sol} za subjekt \texttt{DHL} dobijo informacijo,
da je transporter trenutno označen kot zaupanja vreden (\emph{PASS}) z
agregirano številčno oceno zaupanja približno $6920/10000$ in brez globalnih
opozoril. Če želijo razumeti razloge za razhajanje med pogledi, lahko pregledajo
posamične on-chain zapise orakljev A, B in C in vidijo, da hibridni pogled
Oraklja A subjekt ocenjuje previdneje, medtem ko dokazni in telemetrijski
pogled ocenjujeta, da je transporter glede na preverljive poverilnice in
operativne kazalnike dovolj zaupanja vreden.
\subsection{Scenarij 2: Poslabšanje zaupanja po incidentih transporterja DHL}
\label{subsec:scenarij-2}

V drugem scenariju nadaljujemo primer transporterja \texttt{DHL} iz
scenarija~\ref{subsec:scenarij-1}, vendar predpostavimo, da se v naslednjem
obdobju zgodi več incidentov v dostavni verigi. Namen scenarija je pokazati,
kako sistem dinamično prilagodi oceno zaupanja na podlagi novih podatkov in
kako semantični ter telemetrijski signali lahko pretehtajo formalna dokazila
(VC). Pri tem posebej poudarimo, da bi v realnem okolju ti podatki prihajali
iz zunanjih sistemov, do katerih ima vsak orakelj svoj, ne nujno enak dostop.

\subsubsection*{Spremembe v telemetriji in zunanjih virih}

Simuliramo, da se je v opazovanem obdobju povečalo število poznih dostav in odstopanj v senzorskih meritvah. V prototipu to predstavimo kot spremembo zapisa v datoteki z metrikami telemetrije:

\begin{verbatim}
"http://example.org/trust#DHL": {
  "tempDeviation": 0.12,
  "lateDeliveries": 0.12,
  "alerts": 2
},
\end{verbatim}

V ontologiji in repozitoriju VC subjekt \texttt{DHL} ostane enak kot v
scenariju 1: še vedno ima veljavno preverljivo poverilnico skladnosti in
status transporterja. Spremeni se torej le telemetrija.

V realnem sistemu tak zapis ne bi bil lokalna JSON datoteka, temveč
projekcija zunanjih podatkov:
\begin{itemize}
  \item meritve iz senzorjev IoT (npr. temperatura, vlaga),
  \item zapisi iz transportnega informacijskega sistema (TMS) o zamudah,
  \item opozorila iz nadzornih sistemov ali sistemov za kakovost.
\end{itemize}
Vsak orakelj bi bil konfiguriran z lastnimi adapterji do takih virov. Nekaterim orakljem so ti podatki ključni drugi jih namerno ignorirajo ali jim dajejo zelo majhno težo. V prototipu te razlike modeliramo z različnimi algoritmi orakljev A, B in C nad istim simuliranim telemetrijskim virom.

\subsubsection*{Novi izračuni orakljev}

Na podlagi spremenjenih podatkov oraklji ponovno izvedejo svoje algoritme
in pripravijo nova podpisana poročila. Rezultati so predstavljeni v tabeli \ref{tab:scenarij2-dhl}.

\begin{table}[ht]
  \centering
  \caption{Rezultati orakljev A, B in C ter agregatorja za scenarij 2 (transporter DHL).}
  \label{tab:scenarij2-dhl}
  \begin{tabularx}{\textwidth}{l X c c c}
    \hline
    Vir & Tip algoritma & Odločitev & Številčna ocena & Zastavice \\
    \hline
    Orakelj A & hibridni      & \emph{FAIL} & 2382 & 0x02 \\
    Orakelj B & dokazni (VC)  & \emph{PASS} & 9500 & 0x00 \\
    Orakelj C & telemetrija   & \emph{FAIL} & 6920 & 0x02 \\
    Agregator & agregirano    & \emph{FAIL} & 6267 & 0x02 \\
    \hline
  \end{tabularx}
\end{table}

Interpretacija posameznih poročil:

\begin{itemize}
  \item \textbf{Orakelj A} (hibridni model) ostane negativen (\emph{FAIL}) z
        nizko oceno $2382$. Zastavica \texttt{0x02} označuje, da je ocena
        pod notranjim pragom (\emph{nizka ocena}). V realnem okolju bi ta
        orakelj poleg ontologije in VC bral tudi izbran nabor zunanjih
        podatkov (npr.\ regulatorne incidente, povzetke revizij, agregirane
        metrike kakovosti) in jih semantično povezal v končno oceno.
  \item \textbf{Orakelj B} (dokazni, VC model) se opira predvsem na
        preverljivo poverilnico, ki je še vedno veljavna. Odločitev ostane
        pozitivna (\emph{PASS}) z zelo visoko oceno $9500$ in brez
        zastavic (\texttt{0x00}). V realnem scenariju bi ta orakelj tipično
        bral samo iz infrastrukture za digitalna dokazila (npr.\ veriga
        blokov z VC, revokacijski registri) in ne bi bil neposredno
        povezan z operativno telemetrijo.
  \item \textbf{Orakelj C} (telemetrija) neposredno zazna poslabšanje
        operativnih metrik. Pri novih vrednostih
        \texttt{lateDeliveries} $= 0{,}12$ in \texttt{tempDeviation} $= 0{,}12$
        so \emph{točnost} in \emph{temperaturna stabilnost} le še približno
        $0{,}88$, število alarmov pa se poveča na $2$. Po vgrajeni formuli
        za oceno
        \[
          \text{base\_score} =
           0{,}6 \cdot 0{,}88 + 0{,}3 \cdot 0{,}88 - 0{,}05 \cdot 2 \approx 0{,}692,
        \]
        kar daje številčno oceno $6920/10000$. Ker je ocena
        pod pragom in je število alarmov $\geq 2$, orakelj C spremeni
        odločitev v \emph{FAIL} in nastavi zastavico \texttt{0x02}
        (nizka ocena). V realnem sistemu bi ta orakelj poslušal neposredno tokove telemetrije in iz njih sproti gradil
        oceno zaupanja.
\end{itemize}

\subsubsection*{Agregirana odločitev po incidentih}

Agregator v tem scenariju združi poročila tako kot v
scenariju~\ref{subsec:scenarij-1}: uporabi večinsko glasovanje za diskretno
odločitev, povprečje za številčno oceno ter večinsko pravilo za zastavice.

\begin{itemize}
  \item \textbf{Diskretna odločitev.} Orakelj A in C poročata
        \emph{FAIL}, Orakelj B poroča \emph{PASS}. Dva od treh orakljev sta
        negativna, zato je agregirana odločitev \emph{FAIL}. Z vidika
        sistema je \texttt{DHL} po incidentih označen kot nezaupanja vreden.
  \item \textbf{Številčna ocena.} Ocene treh orakljev so $2382$, $9500$ in
        $6920$. Aritmetična sredina je približno $6267$, kar je nova
        agregirana številčna ocena zaupanja:
        \[
          \frac{2382 + 9500 + 6920}{3} \approx 6267.
        \]
        Ocena je bistveno nižja kot v prvem scenariju (kjer je znašala
        približno $6920$), kar odraža vpliv poslabšanih operativnih metrik.
  \item \textbf{Zastavice} Oraklja A in C nastavljata zastavico
        \texttt{0x02} (\emph{nizka ocena}), Orakelj B pa zastavic ne
        nastavlja. Ker ima ta bit nastavljena večina orakljev, agregator
        v končnem rezultatu prav tako nastavi \texttt{0x02}. Agregirani
        zapis tako ne le označi subjekt kot nezaupanja vreden, temveč
        tudi razloži razlog: nizka številčna ocena, pod pragom sprejemljive
        ravni.
  \item \textbf{Politike in poverilnice} Vsa tri poročila uporabljajo enako zgoščeno vrednost poverilnic \texttt{credential\_hash}, kar pomeni, da se še vedno sklicujejo na isto veljavno preverljivo poverilnico. Agregator pri tem preveri, da poverilnica ni označena kot preklicana. To pomeni, da je sprememba ocene iz \emph{PASS} (scenarij 1) v \emph{FAIL} (scenarij 2) posledica izključno poslabšanja operativnih metrik in zunanjih podatkov, ne pa spremembe formalnih dokazil.
\end{itemize}

Za ostale akterje v dobavni verigi je rezultat scenarija naslednji: ob
poizvedbi nad \texttt{TrustGraph.sol} za subjekt \texttt{DHL} dobijo informacijo,
da je transporter po novih incidentih označen kot nezaupanja vreden
(\emph{FAIL}) z agregirano številčno oceno zaupanja približno $6267/10000$ in z zastavico \texttt{0x02}, ki označuje nizko oceno. Primerjava scenarijev~\ref{subsec:scenarij-1} in~\ref{subsec:scenarij-2} jasno pokaže, da je sistem dinamično prilagodil oceno zaupanja na podlagi novih podatkov, pri čemer so semantični in telemetrijski signali pretehtali formalna dokazila, ki jih je predstavljala preverljiva poverilnica.

\section{Varnostna analiza sistema}
\label{sec:varnostna-analiza}

V tem poglavju obravnavamo varnostni vidik ključnega dela razvitega sistema – decentraliziranega omrežja orakljev (DON) in pametne pogodbe. Prav ti dve komponenti skupaj izvajata proces izračuna zaupanja, pripravljata poročila, preverjata njihove podpise in na koncu zapišeta agregirane odločitve v nespremenljivo podatkovno strukturo na verigi.

Ostali deli arhitekture, kot so podatkovna baza preverljivih poverilnic, ontologija v strežniku Fuseki ali lokalni telemetrični viri podatkov, niso predmet podrobne varnostne analize v tem poglavju. Te komponente obravnavamo kot zunanje storitve, ki imajo svoje mehanizme za zagotavljanje integritete, razpoložljivosti in zanesljivosti. Varnostno pomembne so predvsem v toliko, kolikor lahko s svojimi podatki vplivajo na vhodne informacije za oraklje, ne vplivajo pa neposredno na zapisovanje ali branje rezultatov na verigi blokov.

Osrednji del analize zato posvečamo mehanizmom, ki odločajo o tem, komu system zaupa pri izračunu in zapisu rezultatov, ter kako se podatki premikajo med posameznimi komponentami. Ključni poudarek je na dveh vidikih: prvič, na zaščiti pametne pogodbe pred nepooblaščenimi vpisi ali manipulacijami, in drugič, na zagotovilih, da lahko agregator prepozna veljavne prispevke orakljev, preveri njihove podpise in zavrne kakršne koli nepravilne ali ponarejene podatke. Tok podatkov med oraklji in agregatorjem je zato ena izmed osrednjih točk varnostne presoje, saj se prav tam zgodi največji del logike, ki odloča, katera poročila bodo na koncu vplivala na stanje sistema.

V nadaljevanju najprej opišemo metodologijo in model napadalca, ki ga uporabljamo v analizi, nato pa sistematično pregledamo grožnje, ki zadevajo pametno pogodbo in decentralizirano omrežje orakljev, ter prikažemo obstoječe in predlagane mehanizme za njihovo ublažitev.

\subsection{Cilji varnostne analize}
\label{subsec:cilji-varnost}

Prvi cilj varnostne analize je konceptualen: želimo jasno opredeliti, katere varnostne lastnosti mora imeti sistem, ki združuje verigo blokov, pametne pogodbe in decentralizirano omrežje orakljev. Pri takem sistemu mora biti zagotovljeno, da so odločitve o zaupanju, zapisane na verigi, pravilne, preverljive in odporne na manipulacije, ter da jih lahko drugi akterji neodvisno preverijo. Sistem mora vzdržati tudi primere nepoštenih ali kompromitiranih orakljev, napake infrastrukture in poskuse vplivanja na agregirane rezultate.

Drugi cilj je praktičen in izhaja iz konkretne implementacije prototipa. Pregledati moramo, katere od teh ključnih lastnosti so že ustrezno podprte in kje še obstajajo vrzeli, ki bi jih bilo treba zapolniti za namestitev v produkcijsko okolje. Pri tem se osredotočamo na integriteto podatkov, zapisano v pametni pogodbi, na avtentikacijo orakljev in agregatorja, na preverljivost posameznih poročil ter na razpoložljivost mehanizma za izračun zaupanja, tudi kadar pride do izpada ali nepravilnega delovanja posameznih komponent.

Varnostna analiza se tako osredotoča na del sistema, kjer se sprejemajo in zapisujejo odločitve o zaupanju. Posebej nas zanima, ali je potek podatkov med oraklji, agregatorjem in pametno pogodbo zaščiten pred napačnimi podpisi, ponarejenimi identitetami, spreminjanjem vsebine poročil, napadi zavrnitve storitve in drugimi oblikami poskusov vplivanja na rezultat. Ker so prav ti podatki podlaga za nadaljnje odločitve v dobavni verigi, mora biti jedro sistema načrtovano tako, da nepravilne ali zlonamerne posege zazna in zavrne ter omogoča preverljivo delovanje.

\subsection{Pregled arhitekture DON in pametne pogodbe}
\label{subsec:pregled-arhitekture-varnost}

Za potrebe varnostne analize na kratko povzamemo del arhitekture, ki je ključen za mehanizem zaupanja. Pametna pogodba \texttt{TrustGraph.sol}, ki teče na testnem Ethereum omrežju, hrani agregirane odločitve o zaupanju za posamezne subjekte ter posamične prispevke orakljev, vezane na identifikatorje zahtev \texttt{requestId}. Agregator deluje kot edini pooblaščeni zapisovalec teh rezultatov in ima zato lasten naslov ter zasebni ključ v omrežju. Klice na pametno pogodbo \texttt{TrustGraph.sol} pošilja prek varnega klica RPC do vozlišča verige blokov, pri čemer je identiteta klicatelja na ravni Ethereum protokola kriptografsko potrjena s podpisom transakcije.

Funkcije pametne pogodbe, ki sprejemajo rezultate, so zaščitene z vlogo oziroma modifikatorjem \texttt{onlyEvaluator}, kar pomeni, da jih lahko kliče le z zasebnim ključem, katerega lastnik je agregator. Klici z drugih naslovov povzročijo zavrnitev transakcije (v pametnih pogodbah se kliče funkcija \texttt{revert}). Za spremembe nastavitev, kot so menjava agregatorja ali dodeljevanje novih vlog, je predvideno upravljanje z vlogami in potrjevanjem z več podpisi(angl.\ \emph{multi-sig}). To preprečuje, da bi en sam zasebni ključ samovoljno spremenil pravila delovanja sistema. 

Na verigo se nikoli ne pošiljajo surovi vhodni podatki. Zapisujejo se le zgoščene vrednosti (povzetki politike, preverljivega dokazila in podatki iz ontologije), agregirana ocena ter opozorilne zastavice. Pametna pogodba tako deluje kot kriptografsko zavarovana osrednja plast, ki zagotavlja integriteto rezultatov, ne da bi hranila osebne ali poslovno občutljive podatke.

Oraklji spremljajo dogodke \texttt{TrustOracleRequested}, pridobijo vhodne podatke iz plasti znanja ter izračunajo svojo oceno zaupanja. Ko je izračun opravljen, orakelj pripravi podpisano poročilo \texttt{SignedOracleReport} in ga posreduje agregatorju prek klica HTTP \texttt{/reports}. Vsak orakelj ima lasten par javnega in zasebnega ključa, ki ga uporablja za digitalno podpisovanje poročil s shemo ECDSA \texttt{secp256k1}. Pred podpisom se celotno poročilo pretvori v JSON predstavitev, kjer sta vrstni red ključev in oblika serializacije vnaprej določena. Ta pristop zagotavlja, da ima identičen podatkovni objekt vedno enako binarno predstavitev in s tem tudi enoličen kriptografski podpis, kar je bistveno za preverljivost.

Agregator ob prejemu poročila najprej izvede matematično obnovo javnega ključa iz podpisa (t. i.\ ECDSA \emph{public key recovery}) in iz njega izpelje Ethereum naslov pošiljatelja. Na ta način preveri, ali podpis res ustreza podatkom v poročilu in ali naslov pripada oraklju, ki je na seznamu dovoljenih vozlišč. Če se podpis ne ujema ali če naslov ni dovoljen, agregator poročilo zavrne. Tako sistem ne temelji na zaupanju v omrežno povezavo, temveč na kriptografski verifikaciji vsake posamične oddaje.

Pri branju iz verige blokov pametna pogodba ponuja standardne funkcije tipa \texttt{view} in \texttt{pure}. Prek njih lahko katerikoli odjemalec brez plačila za izvajanje (\texttt{gas}) dostopa do trenutnih in zgodovinskih rezultatov ocenjevanja. Dodatni podatki so na voljo tudi prek dnevnikov dogodkov, ki jih objavi pogodba ob zapisovanju posamičnih prispevkov in končnih agregiranih rezultatov. Ker se na verigi hranijo le zgoščene vrednosti in psevdonimni identifikatorji, branje podatkov ne razkriva nobenih surovih informacij o subjektih ali njihovih lastnostih.

Takšen podatkovni tok določa glavne meje zaupanja in razkriva ključne točke, kjer je sistem potencialno ranljiv: pri klicih pametne pogodbe (kdo lahko sproži izračun in kdo lahko objavi rezultat), na povezavi med pametno pogodbo in oraklji (pravilna interpretacija identitete in politike) ter na povezavi med oraklji in agregatorjem (preverjanje pristnosti pošiljateljev, zaščita pred ponarejenimi podpisi, preverjanje preverljivih dokazil ter zaščita pred napadi ponovnega predvajanja). V nadaljevanju zato ločeno obravnavamo grožnje za pametno pogodbo \texttt{TrustGraph.sol} in grožnje za decentralizirano omrežje orakljev, ter pokažemo, kateri mehanizmi so že implementirani in kje bi bilo treba zasnovo dodatno utrditi pri prehodu v produkcijsko okolje.

\subsection{Metodologija analize in model napadalca}
\label{subsec:metodologija-napadalec}

Varnostno analizo izvedemo po načelih formalnega modeliranja groženj, kot jih predlagajo smernice OWASP in Microsoftovi standardi za decentralizirane sisteme. Osrednji okvir, ki ga uporabimo, je model STRIDE, ki razvršča varnostne grožnje v šest razredov glede na vrsto kršitve: lažno predstavljanje identitete (Spoofing), spreminjanje podatkov (Tampering), zanikanje sodelovanja (Repudiation), razkritje informacij (Information disclosure), zavrnitev storitve (Denial of service) ter povečanje privilegijev (Elevation of privilege). STRIDE je primeren za naš sistem, ker se osredotoča na integriteto, avtentičnost in dostopnost podatkov — lastnosti, ki so v mehanizmu zaupanja ključne, saj neposredno vplivajo na pravilnost izračunov in na zanesljivost registra zaupanja v pametni pogodbi.

Pri analizi ne razčlenjujemo le posameznih komponent, temveč tudi meje med njimi. V sistemu so te meje izrazite: pametna pogodba predstavlja kriptografsko zaščiteno točko trajnega dnevnika, oraklji izvajajo izračune zunaj verige in so zato izpostavljeni manipulacijam, agregator pa deluje kot povezovalni element med izračuni in verigo blokov. STRIDE zato uporabimo kot okvir skozi celoten tok podatkov. Na grožnje se osredotočamo tam, kjer so povezave najšibkejše: pri podpisovanju poročil, pri prenosu rezultatov v agregator, pri preverjanju upravičenosti klicev pametne pogodbe ter pri mehanizmih, ki določajo, kdo lahko objavi končno oceno na verigi.

Čeprav je STRIDE široko uporabljen model, se je treba zavedati njegovih omejitev. Ne obravnava vprašanj gospodarskih spodbud, semantičnih napadov ali zlorab na ravni ontologije, zato te grožnje obravnavamo ločeno kot dopolnilno analizo.

Pri določanju tipa napadalca predpostavimo več ravni zmogljivosti. Prvi scenarij predstavlja zunanjega napadalca, ki ima enaka dovoljenja kot tipični uporabniki verige blokov: lahko kliče javne funkcije pametne pogodbe, spremlja dogodke in poskuša  sprožiti izračune ali preobremeniti pogodbo. Drugi scenarij predvideva škodoželjnega oraklja. Tak orakelj lahko odda napačno poročilo, manipulira z vhodnimi podatki ali poskuša ponarediti podpis. Ker uporabljamo digitalne podpise ECDSA nad deterministično JSON predstavitvijo, tak napadalec ne more ponarediti avtentičnosti, lahko pa poskuša vplivati na izračun s prirejenimi podatki iz ontologije ali preverljivih dokazil.

Tretji scenarij predvideva napadalca, ki cilja na spremembo agregiranja. Ta bi lahko poskušal sprejeti lažna poročila, zlorabiti svoj položaj pri zapisovanju rezultatov ali poskušal reproducirati stare podpise. Ker agregator rekonstruira naslov pošiljatelja iz digitalnega podpisa in preverja časovni žig ter identifikator izvajanja, je zaščiten pred ponarejenimi sporočili in ponavljanjem že obdelanih poročil. Zadnja kategorija so napadi zavrnitve storitve, kjer napadalec poskuša preobremeniti oraklje ali agregator z velikim številom zahtev in tako zmanjšati razpoložljivost izračunov.

Tak model napadalca nam omogoča, da jasno opredelimo, katere komponente se morajo zanašati na kriptografske zaščite (podpisi, zgoščene vrednosti, preverjanje naslovov), katere so zaščitene z logiko pametne pogodbe (vloge, omejitve klicev) ter kje sistem zahteva dodatne arhitekturne ukrepe za produkcijsko okolje, kot so redundančni agregatorji, nadzor nad ključi ter mehanizmi zaščite pred preobremenitvijo. STRIDE tako ponuja ogrodje, ki ga v nadaljevanju uporabimo za razvrščanje groženj in vrednotenje implementiranih protiukrepov.

\subsection{Analiza groženj za pametno pogodbo}
\label{subsec:groznje-contract}

Pametna pogodba \texttt{TrustGraph.sol} predstavlja edini vir resnice sistema na verigi blokov, saj hrani končne odločitve o zaupanju in posamične prispevke orakljev. Kot smo objasnili v prejšnjem podpoglavju, je pomembno, da sprejema le veljavne agregirane rezultate, pravilno povezuje prispevke z identifikatorji zahtev \texttt{requestId} in zavrača nepooblaščene spremembe stanj. V tabeli~\ref{tab:threat-contract} je zbran pregled glavnih razredov groženj (po modelu STRIDE), tipičnih protiukrepov ter oblik testnega pokrivanja.

\begin{landscape}
\begin{table}[ht]
  \centering
  \caption{Povzetek glavnih groženj in protiukrepov za pametno pogodbo \texttt{TrustGraph.sol}.}
  \label{tab:threat-contract}
  \begin{tabularx}{\linewidth}{|X|X|X|}
    \hline
    Grožnja & Protiukrepi & Testno pokritje \\
    \hline
    Nepooblaščeni klici funkcij za zapis rezultatov & Vloga \texttt{EVALUATOR} oziroma modifikator \texttt{onlyEvaluator}, ki omeji klice na naslov(e) agregatorja & Enotski testi klicev z nedovoljenih naslovov, ki pričakujejo \texttt{revert} \\
    \hline
    Neveljavne spremembe stanja odločitev o zaupanju & Preverjanje parametrov in invarianti stanja (ujemanje \texttt{requestId}, monotoni časovni žigi, veljavne zastavice) & Fuzzing in regresijski testi zaporednih klicev, ki preverjajo, da pogodba ne ostane v nelogičnem stanju \\
    \hline
    Preobremenitev pogodbe z velikim številom zahtev ali dragimi klici & Omejitve, kdo lahko sproži nove zahteve, ter preverjanje velikosti in kompleksnosti zapisov & Merjenje porabe plina in testi mejnih primerov za funkcije \texttt{requestTrustReport} in \texttt{fulfillTrustReport} \\
    \hline
    Prevelika količina podatkov o subjektih na verigi & Shranjevanje le zgoščenih identifikatorjev (hash VC, hash politike, hash ontologije) in agregiranih ocen & Pregled sheme stanja in testi za nove različice, ki preverjajo, da se na verigo ne dodajajo surovi identifikatorji ali PII \\
    \hline
    Zloraba upravljavskih oziroma administrativnih vlog & Vloge in večpodpisno upravljanje nad kritičnimi konfiguracijskimi funkcijami (npr.\ zamenjava agregatorja) & Enotski testi dodeljevanja, rotacije in preklica vlog ter zaščite privilegiranih funkcij \\
    \hline
  \end{tabularx}
\end{table}
\end{landscape}

Kot prikazuje tabela~\ref{tab:threat-contract}, je prvi sklop groženj povezan z nepooblaščenimi klici funkcij, ki spreminjajo stanje grafa zaupanja (angl. spoofing). V implementaciji funkcije za zapis rezultatov klice omejujemo z vlogo \texttt{EVALUATOR} oziroma modifikatorjem \texttt{onlyEvaluator}, tako da jih lahko kliče le agregator oziroma izrecno pooblaščeni naslovi. Enotski Foundry testi simulirajo klice z drugih naslovov in preverjajo, da pogodba takšne poskuse zavrne z \texttt{revert}.

Drugi sklop groženj se nanaša na integriteto stanja pametne pogodbe (angl. tampering). Čeprav veriga blokov zagotavlja nespremenljivost zgodovine transakcij, to samo po sebi ne prepreči, da pametna pogodba zaradi neustrezno preverjenih parametrov ali pomanjkljivih logičnih pogojev vnese v svoje stanje podatke, ki so med seboj nekonsistentni. Do takšnih razmer lahko pride denimo takrat, ko se nov rezultat napačno poveže z drugim subjektom, ali ko se časovni žig izračuna v preteklost. To iz vidika semantike predstavlja nelogično stanje. 
Zato so v implementaciji funkcij uvedeni preverjevalni mehanizmi, ki zagotavljajo ohranjanje osnovnih stanjskih invarint (angl. state invariants). Med njih spadajo: zahteva, da se vsak novi zapis nanaša na isti identifikator subjekta kot začetni zahtevek, da je časovni žig (\texttt{asOf}) monoton in nikoli ne nazaduje ter da so meje dovoljenih vrednosti (npr.\ razpon ocene) ustrezno nadzorovane.
Za dodatno zagotovitev pravilnosti smo izvedli kombinacijo enotskih in regresijskih testov. Enotski testi preverjajo posamezne robne primere. Fuzzing testi pa avtomatsko generirajo veliko število naključnih kombinacij parametrov (različni \texttt{requestId}, časovni žigi, zastavice in ocene). S tem preverimo, ali bi katera od kombinacij lahko pripeljala pogodbo v nelogično stanje ali sprožila neželene stranske učinke. Regresijski testi pa zagotavljajo, da popravki logike ne uvajajo novih nepravilnosti in da se znane invariante v kasnejših posodobitvah ohranjajo.

Tretji sklop se nanaša na napade zavrnitve storitve (DoS). Ker so nekatere funkcije pametne pogodbe javno dostopne, lahko velik del zahtev za izračun zaupanja ali nepravilno konfigurirani paketni klici povzročijo, da postane uporaba pogodbe ekonomsko neupravičena. Pri prehodu v verigo bi bilo treba omejiti, kdo lahko sproži \texttt{requestTrustReport}, uvesti pragove za velikost paketov ter spremljati porabo stroškov verige blokov pri tipičnih scenarijih, da se pravočasno zazna degradacija.

Četrti sklop pokriva razkritje informacij (angl. information disclosure). Pametna pogodba shranjuje le zgoščene identifikatorje subjektov in politik, agregirano številčno oceno zaupanja ter zastavice, kar sledi načelu minimizacije podatkov. Pri nadaljnjem razvoju je pomembno ohraniti isto filozofijo tudi pri razširitvah sheme: novi podatki naj bodo vedno agregirani ali zgoščeni, enotski testi pa naj preverjajo, da pogodba ne začne nenamerno shranjevati podrobnih metrik ali neposrednih identifikatorjev podjetij in pacientov.

Zadnji sklop se nanaša na zlorabo privilegijev (angl. elevation of privilege). V produkcijskem okolju pričakujemo upravljavske funkcije za posodobitev konfiguracije ali migracijo stanja (npr.\ zamenjava agregatorja, posodobitev parametrov politik). Take funkcije je treba zaščititi z uveljavljenimi vzorci, kot so vloge in večpodpisno upravljanje, ter podpreti z enotskimi testi, ki preverjajo pravilno dodeljevanje, rotacijo in preklic vlog ter varnostne meje med njimi.

\subsection{Analiza groženj za decentralizirano omrežje orakljev}
\label{subsec:groznje-don}

Mreža orakljev in agregator skupaj tvorita decentralizirano omrežje, ki je odgovorno za izračun in agregacijo ocen zaupanja. Pretok podatkov poteka tako, da pametna pogodba izda dogodek \texttt{TrustOracleRequested} za dani \texttt{requestId} in subjekt, oraklji preberejo dogodek, iz zunanjih podatkov izračunajo oceno zaupanja in pripravijo podpisano poročilo, agregator pa sprejme poročila prek vmesnika HTTP \texttt{/reports}, preveri podpise, preveri priložene poverilnice, izvede agregacijo in rezultat zapiše v pametno pogodbo. Tabela~\ref{tab:threat-don} povzema glavne grožnje po modelu STRIDE, tipične protiukrepe ter predvideno testno pokritje na ravni orakljev in agregatorja.

\begin{landscape}
\begin{table}[ht]
  \centering
  \caption{Povzetek glavnih groženj in protiukrepov za mrežo orakljev in agregator.}
  \label{tab:threat-don}
  \small
  \begin{tabularx}{\linewidth}{|X|X|X|}
    \hline
    \textbf{Grožnja} & \textbf{Protiukrepi} & \textbf{Testno pokritje} \\
    \hline
    Lažni orakelj pošilja poročila agregatorju 
      & Kriptografsko preverjanje podpisov (ECDSA \texttt{recover}) in seznam dovoljenih naslovov orakljev 
      & Integracijski test, ki na \texttt{/reports} pošlje napačno podpisan zahtevek ali zahtevek z neznanega naslova in pričakuje napako \\
    \hline
    Spreminjanje vsebine poročil med prenosom ali v agregatorju 
      & Podpis kanonične JSON predstavitve, preverjanje hashov in zavrnitev pri neskladju 
      & Enotski testi deterministične serializacije ter preverjanja podpisov in hashov \\
    \hline
    Orakelj zanika, da je oddal določeno poročilo 
      & Shranjevanje podpisanih poročil oziroma njihovih hashov ter deterministična obnova naslova iz podpisa 
      & Testi obnove naslova iz podpisa in ujemanja z zapisanim \texttt{node\_id} ter revizijske sledi \\
    \hline
    Prekomerno logiranje identitet subjektov ali poverilnic 
      & Redakcija logov (hashi namesto URI-jev) in uporaba psevdonimnih identifikatorjev 
      & Enotski testi nad vzorčnimi logi, ki preverjajo, da se občutljivi podatki ne pojavijo v izpisu \\
    \hline
    Preobremenitev orakljev ali agregatorja z zahtevki 
      & Omejitve na nivoju omrežja (rate limiting, požarni zid), kvote in redundanca vozlišč 
      & Obremenitveni testi HTTP vmesnika ter spremljanje odzivnega časa in števila zahtevkov \\
    \hline
    Kompromitiran agregator zlorabi svojo vlogo pri zapisovanju rezultatov 
      & Več agregatorjev ali delna agregacija na verigi, ločeno upravljanje konfiguracije in jasna omejitev pooblastil 
      & Simulacijski testi različnih konfiguracij kvora in scenarijev kompromitiranih vozlišč \\
    \bottomrule
  \end{tabularx}
\end{table}
\end{landscape}


Prva skupina groženj za DON je povezana z istovetnostjo orakljev (spoofing). Agregator pri prejemu poročila kriptografsko obnovi naslov iz podpisa in ga preveri proti konfiguriranemu seznamu dovoljenih vozlišč; poročila z neznanih naslovov ali napačnimi podpisi zavrne. Ker podpis vključuje tudi nonce oziroma \emph{run-id} in časovni žig, lahko agregator zazna in zavrne ponovno predvajanje starega sporočila in s tem prepreči replay napade. Integracijski testi za HTTP vmesnik preverijo, da so spremenjeni ali napačno podpisani payloadi zavrnjeni z ustrezno napako.

Pri tamperingu ščitimo predvsem integriteto vsebine poročil. Ker je podpisan kanonični JSON celotnega poročila, agregator zazna vsako spremembo vsebine med prenosom ali vmesnim shranjevanjem. Za produkcijsko okolje je smiselno dodati revizijsko sled (na primer zapis hashov sprejetih poročil), kar omogoča naknadno preverjanje skladnosti med prejetimi poročili in on-chain zapisi. Enotski testi pri tem preverijo, da serializacija poročila v kanonično obliko vedno vrne enak rezultat in da vsaka sprememba vsebine povzroči neuspeh preverjanja podpisa.

Grožnja repudiacije je pomembna z vidika transparentnosti: orakelj bi lahko poskušal zanikati, da je oddal negativno poročilo za določenega partnerja. S podpisanimi poročili in deterministično obnovo naslova iz podpisa lahko vedno pokažemo, kateri ključ je poročilo podpisal. Če agregator dodatno hrani hash sprejetih poročil, je revizijska sled za potrebe kasnejših sporov ali regulatornih pregledov relativno preprosto dosegljiva.

Poseben del zasnove DON je tudi preverjanje preverljivih poverilnic (VC), ki spremljajo poročila. Vsak orakelj poleg ocene zaupanja priloži VC, ki dokazuje njegovo pooblastilo ali lastnosti subjekta; agregator pred sprejemom poročila preveri podpis VC, status preklica, časovno veljavnost in skladnost sheme z vnaprej definirano politiko. Poročila z neveljavnimi ali preklicanimi poverilnicami so zavrnjena in se nikoli ne zapišejo na verigo, kar zmanjša tveganje, da bi se na grafe zaupanja oprli ponarejeni ali zastareli dokazi.

Napadi zavrnitve storitve (DoS) ciljajo na razpoložljivost omrežja. Ker v prototipu omejevanje hitrosti ni povsod implementirano, so oraklji in agregator teoretično dovzetni za preobremenitev z veliko količino zahtev. V produkcijski rabi bi morali definirati kvote na nivoju omrežja (API prehodi, požarni zidovi), dodati health-checke ter zagotoviti redundanco vozlišč, tako da izpad posameznega oraklja ali agregatorja ne onemogoči sistema in da se promet enakomerno porazdeli.

Informacijsko razkritje v DON se tipično zgodi skozi dnevnike. Ker oraklji in agregator redno logirajo stanje in dogodke, obstaja nevarnost, da bi v logih ostali neposredni identifikatorji subjektov ali poverilnic. Zato je smiselno vpeljati redakcijo logov (na primer izpis le skrajšanega hash-a namesto celotnega URI-ja) in z enotskimi testi preverjati, da vzorčni logi ne vsebujejo občutljivih podatkov. S tem dopolnimo načelo minimizacije podatkov, ki ga že upoštevamo pri zapisih na verigo.

Zadnji sklop groženj se nanaša na privilegije agregatorja. Trenutni model predvideva en osrednji agregator, ki ima pooblastilo za zapis agregiranih odločitev v pametno pogodbo. Če bi bil ta agregator kompromitiran, bi lahko v imenu legitimnih orakljev pošiljal napačne podatke ali spreminjal rezultat agregacije, preden se ta zapiše na verigo. Pri razširitvi v produkcijo bi bilo smiselno razmisliti o več agregatorjih, ki bi morali doseči soglasje, ali o prenosu dela agregacijske logike v pametno pogodbo samo, tako da posamezno vozlišče ne bi moglo samostojno manipulirati z rezultati. Simulacije različnih konfiguracij kvora in scenarijev kompromitiranih vozlišč lahko pomagajo oceniti, kako odporen bi bil tak sistem na zajetje ali okvaro posameznih komponent.

\subsection{Povzetek tveganj in omejitve prototipa}
\label{subsec:povzetek-don}

V analizi smo se osredotočili na del sistema, ki deluje na verigi blokov in v mreži orakljev. Implementacija že vključuje več ključnih varnostnih mehanizmov: podpisana poročila orakljev, preverjanje naslovov v agregatorju, uporabo nonce in časovnih žigov za zaščito pred replay napadi, osnovno preverjanje preverljivih poverilnic, shranjevanje le zgoščenih identifikatorjev in agregiranih ocen na verigi ter jasno ločitev med on-chain in off-chain logiko.

Kljub temu za uporabo v produkcijskem okolju ostaja odprtih več pomembnih vprašanj: registracija in upravljanje identitet orakljev ter agregatorjev na nivoju pametne pogodbe, varno upravljanje z zasebnimi ključi (shramba, rotacija, odziv ob kompromitaciji), odpornost na napade zavrnitve storitve v DON (redundantni oraklji, več agregatorjev, mehanizmi soglasja) ter dodatne invariante in obsežnejše testiranje pametne pogodbe za preprečevanje nelogičnih stanj. Varnostna analiza tako ne služi le kot opis trenutnega stanja, temveč tudi kot usmeritev, katera področja bi bilo treba prioritetno nadgraditi pri prenosu rešitve v realno, produkcijsko okolje.

\section{Razprava}

\section{Pot do produkcijskega sistema}
\label{sec:pot-do-produkcije}

Trenutna implementacija deluje kot laboratorijska postavitev, v kateri teče omejeno število orakljev, en agregator in ena instanca pametne pogodbe na testnem omrežju verige blokov, vse pa je vezano na poenostavljeno infrastrukturo Docker. Takšen prototip je primeren za demonstracijo koncepta in ponovljivih scenarijev, ni pa neposredno prenosljiv v okolje, kjer bi morali upoštevati zahtevnejše zahteve glede razpoložljivosti, upravljanja identitet, nadzora nad konfiguracijo in dolgoročne vzdržnosti.

V produkcijskem sistemu bi moral DON upravljati večje število neodvisnih orakljev, ki bi jih upravljali različni akterji v dobavni verigi ali specializirani ponudniki storitev zaupanja. Regulator (na primer EMA ali nacionalna agencija) bi imel v takem modelu vlogo skrbnika nad omrežjem orakljev: skrbel bi za upravljanje registracije orakljev, odobritev novih članov, izključevanje neustreznih ter pregled nad konfiguracijo politik. Za to bi bil potreben namenski upravljavski vmesnik, ki bi regulatorju omogočal pregled stanja DON, upravljanje s seznami pooblaščenih vozlišč in konfiguracijo pragov zaupanja brez neposrednega poseganja v kodo pametne pogodbe.

Pomemben del prehoda v produkcijo je formalizacija upravljanja identitet in poverilnic akterjev. Za plast preverljivih poverilnic bi bila naravna izbira uporaba evropske infrastrukture EBSI, ki že definira standardizirane formate VC in DID ter infrastrukturo za preverjanje in preklic. V takem scenariju bi oraklji in akterji v dobavni verigi nastopali z did-identitetami, veriga blokov in DON pa bi se zanašala na EBSI kot na zunanji vir resnice za status poverilnic. Vse, kar bi se zapisalo na verigo, bi bili le povzetki teh VC in politik, medtem ko bi se polna vsebina in osebni podatki obdelovali v okviru EBSI in povezanih sistemov.

Za samo infrastrukturo DON bi bilo smiselno obstoječe rešitve za decentralizirana omrežja orakljev, kot je Chainlink. Ta platforma že nudi vzorce za varno izvajanje orakljev v heterogenem omrežju, mehanizme za nadzor nad zanesljivostjo vozlišč, podpisane povratne klice in zaščito pred posameznimi kompromitiranimi vozlišči. Namesto lastne implementacije bi lahko mehanizem ocenjevanje zaupanja v prihodnje prevedli v Chainlinkove naloge, kjer bi nadzorni sklop (na primer regulator ali konzorcij akterjev) definiral, kateri oraklji sodelujejo v posamezni nalogi, s kakšnimi utežmi in pod katerimi pogoji.

Na ravni pametne pogodbe bi prehod v produkcijo zahteval utrditev mehanizmov upravljanja in nadgradnje. Ključno je, da pogodba jasno loči vloge (regulator, agregatorji, tehnični skrbniki) in omeji njihova pooblastila z dobro preizkušenimi vzorci, kot so večpodpisne denarnice in formalno definirani postopki nadgradnje. Vsak poseg v konfiguracijo, na primer dodajanje ali odstranjevanje oraklja, zamenjava agregatorja ali sprememba pragov zaupanja, bi moral biti transparenten, sledljiv in odobren s strani vnaprej definiranega kvora udeležencev, ne pa vezan na en sam zasebni ključ.

Poleg identitet in upravljanja vlog bi bilo treba v produkcijskem okolju bistveno več pozornosti nameniti varnemu upravljanju zasebnih ključev ter razpoložljivosti celotne arhitekture. Ključi orakljev in agregatorjev bi morali biti shranjeni v varnih modulih, z dolgoživimi politikami rotacije ter jasnimi postopki ob zaznani kompromitaciji. Infrastrukturo orakljev bi bilo treba postaviti v redundantnem okolju, z nadzorovanimi posodobitvami, spremljanjem izpada in mehanizmi za avtomatsko preusmeritev prometa v primeru okvare.

Pot od trenutnega prototipa do produkcijskega sistema bi v praksi potekala v več korakih. Najprej bi bilo smiselno utrditi že implementirane varnostne mehanizme na ravni pametne pogodbe in agregatorja, dodati formalizirano upravljanje vlog ter razširiti testno pokritost. Sledila bi faza razširitve DON z več neodvisnimi oraklji in morda več agregatorji, kjer bi se v pilotnem okolju preizkusile različne konfiguracije kvoruma in algoritmi agregacije. V naslednjem koraku bi se sistem integriral z realno infrastrukturo preverljivih poverilnic (EBSI) in z izbranimi operativnimi sistemi akterjev v dobavni verigi, ob tem pa bi se v sodelovanju z regulatorji definirali formalni postopki uporabe rezultatov zaupanja v odločanju o odpošiljanju in licenciranju. Po uspešnem pilotnem obdobju bi sledil popoln prehod v produkcijsko okolje, z ustrezno dokumentacijo, usposabljanjem uporabnikov in vzpostavitvijo postopkov za dolgoročno vzdrževanje in nadgradnje sistema.

\section{Operativni in regulativni vidiki}
\label{sec:operativni-regulativni-vidiki}

Če želimo mehanizem zaupanja prenesti iz eksperimentalnega okolja v dejansko farmacevtsko dobavno verigo, ga moramo umestiti v obstoječe procese kakovosti in regulative. Ocena iz DON ne bi nadomestila regulatorne odgovornosti, temveč bi delovala kot dodatni, standardiziran vhod v postopke kvalifikacije in periodičnega preverjanja partnerjev (proizvajalcev, distributerjev, transporterjev) ter odločanja o odpošiljanju serij. V praksi bi bila integrirana v obstoječe informacijske sisteme (npr.\ sistem za upravljanje dobaviteljev ali modul za izdajo serije) prek vmesnika API: aplikacija bi ob potrebi sprožila poizvedbo, prejela strukturiran odgovor (odločitev, ocena, časovni žig, hash politike in poverilnic) in ga vključila v revizijsko sled ter dokumentacijo kakovosti. Z vidika upravljanja bi bil DON najverjetneje vzpostavljen kot skupna infrastruktura, nad katero bi imel nadzor regulator ali konzorcij regulatorjev in industrije, ki bi skrbel za referenčne politike, pravila za vstop in izključitev orakljev ter revizijo njihovih algoritmov.

Regulativno in z vidika varovanja podatkov je ključna ločitev med plastjo na verigi in izven. Surovi podatki (osebni, operativni, poslovno občutljivi) ostajajo v sistemih podvrženih GDPR, na verigo pa se zapisujejo le zgoščeni identifikatorji subjektov, politik in poverilnic ter agregirane ocene in zastavice. Veriga tako služi predvsem kot objekt integritete in neizbrisljiv dnevnik: v primeru spora lahko regulator ali audit neodvisno preveri, da objavljena ocena in pripadajoče poverilnice od trenutka objave niso bile spremenjene. Za produkcijsko rabo bi moral biti tudi sam mehanizem DON obravnavan kot validiran računalniški sistem z jasno opredeljenimi kritičnimi funkcijami, kontroliranim upravljanjem sprememb ter periodičnim pregledom algoritmov in pravil agregacije.

\section{Širši potencial in vizija uporabe sistema}
\label{sec:sirsi-potencial-vizija}

Čeprav je bila rešitev v tej nalogi konkretizirana na farmacevtski dobavni verigi, je osnovni vzorec precej širši. V jedru ne upravljamo zaupanja v zdravila, temveč v akterje, ki si izmenjujejo podatke in prevzemajo odgovornost v porazdeljenem sistemu: organizacije z različnimi vlogami, pogosto nasprotujočimi si interesi, heterogene podatkovne vire in potrebo po neodvisno preverljivem zapisu o tem, komu in pod kakšnimi pogoji zaupamo. Enak vzorec se pojavi v drugih dobavnih verigah: pri dobavni verigi hrane, elektronskih komponent, oskrbi z energenti, recikliranju materialov ipd. Povsod, kjer so prisotne tri lastnosti: (i) več neodvisnih organizacij, (ii) različni in pogosto nezdružljivi viri podatkov ter (iii) potreba po transparentnem in neodvisno preverljivem zapisu rezultatov.

Z manjšimi prilagoditvami ontologije in politik bi se isti mehanizem lahko uporabljal tudi v pametnih mestih. Namesto proizvajalcev in transporterjev bi se v grafu zaupanja pojavili upravljavci avtobusnih linij, ponudniki deljenih koles in skirojev, operaterji avtonomnih vozil ter občinske službe, ki skrbijo za prometno infrastrukturo. Regulator (na ravni mesta ali države) bi postavljal pravila glede varnosti, zanesljivosti in okoljskih vplivov, oraklji bi nad kombinacijo telemetrije, revizijskih poročil in preverljivih poverilnic izračunavali ocene zaupanja, pametna pogodba pa bi skrbela za to, da so končne odločitve transparentno in nespremenljivo zabeležene. V takem scenariju bi graf zaupanja postal neke vrste prometni "kompas": komu lahko mesto zaupa pri upravljanju kritične mobilnostne infrastrukture in kateri ponudniki izpolnjujejo dogovorjene standarde.

Podobno bi lahko pristop razširili na druge podatkovno intenzivne domene, kjer sodeluje več neodvisnih organizacij: od energetskih skupnosti, kjer gospodinjstva in podjetja izmenjujejo energijo in fleksibilnost, do čezmejnih podatkovnih prostorov, kjer različni akterji delijo podatke o okolju, zdravju ali mobilnosti. V vseh teh primerih bi se farmacevtska ontologija preprosto zamenjala z ustreznim domenskim modelom, preverljive poverilnice bi zamenjale druge oblike digitalnih dokazil (licence, homologacije, certifikati skladnosti), DON pa bi postal mesto, kjer različni regulatorji, neodvisne institucije, certifikacijski organi ali celo skupnosti uporabnikov prispevajo svoj pogled na zanesljivost in integriteto posameznih akterjev.

Rešitev predstavljena v tej nalogi je torej le prvi korak na poti k bolj razpršenim, transparentnim in odpornih ekosistemom zaupanja v digitalni dobi. S kombinacijo kriptografskih mehanizmov, decentraliziranih omrežij in pametnih pogodb lahko ustvarimo sisteme, kjer zaupanje ni več centralizirano v rokah posameznih akterjev, temveč je rezultat sodelovanja, preverjanja in neodvisnega nadzora. Takšni sistemi imajo potencial, da izboljšajo varnost v številnih kritičnih sektorjih družbe.

\chapter{Zaključek in nadaljnje delo}
\label{sec:zakljucek}


\section{Povzetek dela in prispevki}

V magistrski nalogi smo obravnavali problem upravljanja zaupanja v farmacevtskih dobavnih verigah, kjer poteka izmenjava podatkov med številnimi akterji, ki pogosto delujejo z različnimi interesi in v heterogenih podatkovnih okoljih. Obstoječe prakse upravljanja zaupanja temeljijo predvsem na centraliziranih virih, ročnih revizijah in pogodbenih razmerjih. To otežuje, da je sistem transparenten, ponovljiv in onemogoča avtomatizirano preverjanje. Cilj naloge je bil zato razviti konceptualni in arhitekturni model, ki združuje semantično predstavitev domene, mehanizme verige blokov ter decentralizirane izračune za oblikovanje grafa zaupanja med akterji dobavne verige zdravil.

Najprej smo predstavili teoretično ozadje in pregled obstoječih modelov zaupanja ter poudarili vrzeli, ki se pojavljajo pri uporabi teh modelov v farmacevtskem kontekstu. Na podlagi teh izhodišč smo zasnovali konceptualni model, ki akterje, njihove lastnosti in dokazila predstavi s pomočjo ontologije. Temu modelu smo dodali arhitekturo, ki vključuje pametno pogodbo za zapis rezultatov, omrežje decentraliziranih orakljev za izračun zaupanja ter agregator za usklajevanje njihovih odločitev. V implementaciji smo razvili prototip z uporabo tehnologij Docker, Jena Fuseki, preverljivih poverilnic VC in testnega omrežja Ethereum. Postavili smo tri oraklje, ki izvajajo ocenjevanje zaupanja z različnimi kriteriji in pravili. Skozi dva eksperimentalna scenarija smo prikazali, kako sistem obravnava realne spremembe podatkov in kako decentralizirani pogledi vodijo do dinamičnih ter razložljivih ocen zaupanja. Z varnostno analizo smo nazadnje ocenili grožnje, ki zadevajo pametno pogodbo in decentralizirano omrežje orakljev. Poudarili smo tudi potrebne korake za prehod na delovanje v produkcijskem okolju.

Pomen opravljenega dela je predvsem v tem, da združuje področja, ki so bila v literaturi pogosto obravnavana ločeno: semantično modeliranje domene, decentralizirane identitete, kriptografsko preverjanje in porazdeljeno odločanje. S tem naloga zapolni vrzel med teoretičnimi modeli zaupanja in praktičnimi sistemi, ki jih lahko dejansko uporabljajo regulatorji in organizacije. Predlagana arhitektura pokaže, da je mogoče v kompleksnih okoljih, kot je farmacevtska veriga, oblikovati pregledne in preverljive postopke ocenjevanja. S tem pa se olajša zaznavanje tveganj in poveča zanesljivost odločanja v takšnih sistemih.

\section{Nadaljnje delo}
\label{subsec:omejitve-nadaljnje}

Predstavljeni prototip predstavlja dokaz koncepta, ne pa popolne implementacije sistema, primernega za produkcijsko okolje. Nadaljnji razvoj mora zato vključevati razširitev decentraliziranega omrežja orakljev, ki bi jih upravljali dejanski akterji farmacevtske verige, ter uvedbo mehanizmov ekonomskih spodbud in kazni, ki bi zagotovili pošteno udeležbo in odpornost na zlonamerna dejanja. Prav tako je nujna integracija polne infrastrukture za izdajo in preverjanje poverilnic, skladnih s standardi, kot je EBSI, ter vzpostavitev upravljavskega modela (governance), ki določa pravila za upravljanje identitet, rotacijo ključev in sprejemanje sprememb arhitekture.

Ontologija, uporabljena v prototipu, bi morala biti za uporabo v realnih okoljih razširjena z dodatnimi tipi entitet, regulatornimi zahtevami, pogoji skladnosti ter bolj izraznimi pravili, ki omogočajo kompleksnejše metode sklepanja. Varnostna plat sistema zahteva nadaljnje delo na področju robustnosti pametne pogodbe (formalno preverjanje, zaščita pred napadi zavrnitve storitve, zaščita upravljavskih vlog) ter odpornosti decentraliziranega omrežja orakljev (več agregatorjev, mehanizmi soglasja, varno upravljanje ključev). Pomembno je tudi razviti celovite postopke za validacijo sistema kot računalniškega sistema v skladu z regulativnimi zahtevami farmacevtskega sektorja.

Zanimiva smer nadaljnjega razvoja je tudi preizkus zasnove v drugih sektorjih. Pametna mesta, energetika, logistika, kritična infrastruktura in IoT omrežja predstavljajo naravne kandidate, kjer podobno kot v farmacevtskem sektorju obstaja potreba po transparentnem, razložljivem in preverljivem mehanizmu zaupanja. Razširitev arhitekture na te domene bi omogočila oceno univerzalnosti predstavljenega koncepta in njegovo prilagodljivost različnim operativnim zahtevam. S tem bi lahko nastal generičen okvir za vzpostavitev digitalnih ekosistemov zaupanja, ki bi prispevali k varnejšemu in bolj transparentnemu razvoju porazdeljenih sistemov v digitalni družbi.

\appendix
\chapter{Ontologija OWL farmacevtske verige}
\label{app:trust-ontology}
\lstinputlisting[
  language=XML,
  caption={Ontologija OWL za model zaupanja v farmacevtski verigi},
  breaklines=true,
  label={lst:trust-ontology}
]{ontologies/pharma-trust.owl}


%----------------------------------------------------------------
% SLO: bibliografija
% ENG: bibliography
%----------------------------------------------------------------
\bibliographystyle{elsarticle-num}

%----------------------------------------------------------------
% SLO: odkomentiraj za uporabo zunanje datoteke .bib (ne pozabi je potem prevesti!)
% ENG: uncomment to use .bib file (don't forget to compile it!)
%----------------------------------------------------------------
\bibliography{bibliography}

%----------------------------------------------------------------
% SLO: zakomentiraj spodnji del, če uporabljaš zunanjo .bib datoteko
% ENG: comment the part below if using the .bib file
%----------------------------------------------------------------

%\begin{thebibliography}{99}
%\bibitem{Fortnow} L.\ Fortnow, ``Viewpoint: Time for computer science to grow up'',
%{\it Communications of the ACM}, št.\ 52, zv.\ 8, str.\ 33--35, 2009.
%\bibitem{Knuth} D.\ E.\ Knuth, P. Bendix. ``Simple word problems in universal algebras'', v zborniku: Computational Problems in Abstract Algebra (ur. J. Leech), 1970, str. 263--297.
%\bibitem{Lamport} L.\ Lamport. {\it LaTEX: A Document Preparation System}. Addison-Wesley, 1986.
%\bibitem{ubi} O.\ Patashnik (1998) \BibTeX{}ing.
%Dostopno na: \url{http://ftp.univie.ac.at/packages/tex/biblio/bibtex/contrib/doc/btxdoc.pdf}
%\bibitem{licence} licence-cc.pdf. Dostopno na: \url{https://ucilnica.fri.uni-lj.si/course/view.php?id=274}
%\end{thebibliography}

\end{document}
